\section{Literature Review}

\subsection{Mathematical models}
The CPT process describes both the free streaming of the light ions from one collision site to the next, and the collisions experienced by the light ions with the plasma. The general framework that describes this transport process is provided by the linear Boltzmann transport equation (BTE) \cite{Prinja-2010}. The BTE is an integro-differential equation that generally consists of seven independent variables: three in space, two in angular-directions, one in energy, and one in time. In this work, the simplified time-independent, one-dimensional spherical coordinates version of the general BTE is used. Additionally, the BTE collision operator is modified to account for the multi-species collisions that occur in CPT. The resulting time-independent, multi-species, one-dimensional spherical form of the BTE is referred to as the CPT equation.

Analytical solution methods to the CPT equation are not possible in most applications due to the complexities in the geometry and material cross sections that describe realistic problems. Instead, numerical solution techniques must be used for which two general approaches exist: Monte Carlo methods, and deterministic methods. Monte Carlo methods attempt to solve the CPT equation by using random numbers to track the transport of individual particles in the plasma, and deterministic methods discretize the CPT equation on a geometric mesh into a set of linear equations which are solved using linear algebra techniques. Traditional neutral particle Monte Carlo and deterministic solution methods are inefficient in obtaining accurate solutions to the CPT equation because of the small MFPs that are characteristic of light ions. In Monte Carlo methods, the small MFP causes the method to sample hundreds of thousands of individual collisions per light ion using random numbers. This large volume of random number sampling causes Monte Carlo methods to be very slowly converging. For a deterministic solution method to produce accurate solutions to a transport problem, the spacing between mesh points must be on the same order as the MFP \cite{Larsen-1999}, which in the context of CPT results in a large number of linear equations and unknowns that require significant computational time to solve. 

The inefficiencies exhibited by both Monte Carlo and deterministic solution methods are due to the Boltzmann collision operator, which is an efficient collision operator for the Coulomb collisions that dominate the interaction physics of light ions. Therefore more efficient collision operators are needed to more efficiently represent the interaction physics of light ions. Several collision operators that are more efficient than the Boltzmann collision operator at representing the interaction physics of light ions have been previously developed. The first is the Fokker-Planck (FP) collision operator, which results by retaining the first two terms in a Taylor series expansion of the Boltzmann collision operator about zero angular-deflection and energy-loss \cite{Pomraning-1996}. As a result of the Taylor series expansion, the FP collision operator is highly efficient at modeling problems in which the interaction physics are dominated by zero angular deflection and energy loss collisions. However, the FP collision operator does not account for large-angle scattering collisions which produce recoil ions. 

The second collision operator is the Boltzmann-Fokker-Planck (BFP) collision operator which accounts for the large-angle-scattering collisions by splitting the original Boltzmann collision operator into singular and smooth parts \cite{Ligou-1986}. The singular part describes the small angular-deflection and energy-loss collisions, while the smooth part describes the large-angle scattering collisions. The singular part is then modeled using an FP collision operator, while the smooth part is modeled with a Boltzmann collision operator. The challenge with BFP collision operators is properly splitting the original Boltzmann collision operator such that the peaked part is sufficiently peaked so that the asymptotic limit of the FP scattering is satisfied, and the smooth part is smooth enough that Boltzmann collision operators can be efficiently solved \cite{Pomraning-1992}. 

\subsection{Numerical solution methods}
Several efficient Monte Carlo solution methods have been developed for numerical solutions to the CPT equation, including the condensed history method \cite{Berger-1963}, a hybrid multigroup/continuous energy BFP method \cite{Morel-1996}, and a moment preserving method \cite{Dixon-2015}. However, the focus of this work is on deterministic solution methods and subsequently does not consider the development of efficient Monte Carlo solution methods. 

Several deterministic solution methods have been previously developed. The first is a finite element BFP method developed by Honrubia and Aragones in 1986 \cite{Honrubia-1986}. Honrubia and Aragones developed a fully coupled space-energy-angle linear discontinuous Galerkin discretization for the BFP equation in one-dimensional spherical geometry. The main limitations of that work are: a cutoff angle was used to split the Boltzmann collision operator into peaked and smooth parts, and NES and nuclear-reaction collisions were not taken into account. By using a cutoff angle, a very poor representation of the peaked and smooth parts of the Boltzmann collision operator is obtained \cite{Landesman-1989}. Neglecting NES and nuclear-reaction collisions  means that no recoil ions are produced which have been shown to be a significant source for the heating of the plasma \cite{Nakao-1988}.

The second deterministic method for solving the CPT equation was developed by Nakao, et al. in 1990 \cite{Nakao-1990}. Nakao, et al. developed a fully coupled space-energy one-dimensional spherical linear discontinuous Galerkin BFP solver. Nakao, et al. split the original Boltzmann collision operator by modeling the Coulomb collisions with an FP collision operator, and modeling the NES collision with a Boltzmann collision operator. There were several shortcomings of this work: the first being the failure to include the energy diffusion term in the FP collision operator. Second, better tabulated data that describes the NES and nuclear-reaction collisions is now available \cite{Hale-1983}. 