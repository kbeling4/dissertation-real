\section{Motivation}
Charged particle transport (CPT) is an important problem in a vast array of applications, including: radiotherapy, space radiation, and plasma physics. The application of interest in this work is the transport of energetic light ions in inertial confinement fusion (ICF) plasmas; however, the work presented here can be easily applied to any application in which CPT is important. More specifically, this work focuses on the transport of energetic $\alpha$-particles in a spherical, fully-ionized, thermal deuterium-tritium (DT) plasma. This problem is commonly encountered in ICF applications where DT fusion reactions produce energetic $\alpha$-particles that subsequently transport and deposit energy in the DT plasma \cite{Bellan-2006}. In this work the following assumptions are made: all electric and magnetic effects present in the plasma are neglected, the plasma ions can only become energetic through recoil collisions, and the energetic light ions are absorbed upon reaching the thermal energy of the plasma.

In this proposal the CPT problem is divided into three areas: 1) the interaction physics that describe the collisions that light ions experience in plasma, 2) the mathematical models that describe the transport of the light ions in plasma, and 3) the numerical solution methods for solving the CPT equation. The remainder of this introduction provides brief overviews of the interaction physics, mathematical formulations, current numerical solution methods for the CPT equation, and the research goals of this work.

Energetic light ions interact with the plasma through through three types of collisions: 1) Coulomb collisions, 2) nuclear-elastic scattering (NES) collisions, and 3) nuclear-reaction collisions \cite{Perkins-1981} \cite{Hale-1983}. Coulomb collisions are modeled by the Coulomb potential, which describes the interaction of two point charges that are a distance $r$ apart \cite{Bellan-2006}. Coulomb collisions are characterized by small angular-deflection and energy loss collisions, and small mean-free-paths (MFPs) or average distances between discrete collisions. As an example, for an energetic light ion to slow down to the thermal energy of the plasma via Coulomb collisions, it undergoes many small angular-deflection and energy-loss interactions with the background charged particles present in the plasma. These many small angular-deflection and energy collisions cause the light ions to very nearly continuously be scattered by the charged particles present in the plasma. These almost continuous collision events differ significantly from the discrete collision events experienced by neutral particles. 

NES and nuclear-reaction collisions are very discrete events and are not peaked about zero angular deflection and have large MFPs when compared to Coulomb collisions. The NES collisions are large-angle-elastic scattering collisions that result in significant amounts of energy being transferred from the energetic light ions to the plasma ions creating recoil ions. These recoil ions transport through the plasma creating additional recoil ions until the thermal energy of the plasma is reached. The nuclear-reaction collisions cause the incident ions to be absorbed and create new light ions of different species. These newly created light ions transport through the plasma and create additional recoil ions and cause further nuclear-reactions until reaching the thermal energy of the plasma. This cascading effect resulting from the NES and nuclear-reaction collisions causes the distributions in the plasma of all ion species to be coupled, meaning that the general CPT problem in plasmas is a fully-coupled problem.
