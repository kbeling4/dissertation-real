\section{Objectives and Scope}
The goal of this work is to provide accurate and efficient solution methods for the general CPT transport equation that describes the transport of energetic-light ions in spherical ICF plasmas. To achieve this goal, the current methods for CPT are introduced and research is proposed in all three areas of CPT outlined in this introduction. In the charged particle interaction physics area of CPT, the research that is proposed is the development and proper utilization of analytical and tabulated differential cross sections that describe the interactions of light ions in plasma. This includes the exploration of more accurate Coulomb differential cross sections and screening parameters in addition to the proper utilization of the tabulated NES and nuclear-reaction data by use of two-body collision kinematics.

To improve upon the limitations of the FP and BFP collision operators, the generalized Fokker-Planck (GFP) collision operator is introduced. The GFP collision operator re-normalizes the Taylor series expansion of the Boltzmann collision operator such that a robust set of coupled second order in angle, and first order in energy, differential equations are obtained. As a result of this renormalization, the GFP method preserves the first few terms of the Taylor series expansion of the Boltzmann collision operator exactly and approximates the remaining higher-order moments in terms of the first 2K moments. By retaining these higher-order moments, the GFP collision operator is able to account for larger angular scattering collision than the FP collision operator which completely neglects them.

To accurately solve the CPT equation, the various streaming and collision operators that appear in the CPT equation are discretized using the discrete ordinates \Sn method in angle and an arbitrary order discontinuous Galerkin discretization method in space and energy. These discretizations are presented and verified using the method of manufactured solution in this proposal. Additionally, to numerically solve the CPT equation efficiently, iterative solution methods are proposed for both solving the fully coupled CPT equation and for the transport equation describing each individual ion species.