\section{Transport Preliminaries}
Before proceeding to the introduction of the Boltzmann transport equation in the following section we opt to introduce several key concepts necessary to understand the Boltzmann transport equation. These concepts include: a brief description of the limitations of the Boltzmann transport equation, a definition of the cross sections used by the transport equation, and the independent variables and phase-space of the Boltzmann transport equation.

\subsection{Independent variables}
The solution to the general transport equation gives the distribution of particles in a six-dimensional phase-space at some time $t$. The six independent variables at time $t$ are: position $\vec{r}$, energy $E$, and direction $\hat{\Omega}$. In cartesian coordinates the position with respect to the origin is
\begin{equation}
  \vec{r} = x \, \hat{e}_x + y \, \hat{e}_y + z \, \hat{e}_z,
\end{equation}
where $\hat{e}_x$, $\hat{e}_y$, and $\hat{e}_z$ are mutually orthogonal unit vectors. 

The direction vector $\hat{\Omega}$, a unit vector $(\vert \hat{\Omega} \vert = 1)$, id defined with respect to the polar angle $\theta$, and azimuthal angle $\phi$. In cartesian coordinates $\hat{\Omega}$ is given by
\begin{equation}
  \hat{\Omega} = \Omega_x \hat{e}_i + \Omega_y \, \hat{e}_y + \Omega_z \, \hat{e}_z,
\end{equation}
where
\begin{gather}
  \Omega_x = \sqrt{1 - \cos \theta} \cos \phi, \\
  \Omega_y = \sqrt{1 - \cos \theta} \sin \phi, \\
  \Omega_z = \cos \theta.
\end{gather}

If we independently displace the spatial variables $x$, $y$, and $z$ by incremental amounts $dx$, $dy$, and $dz$, the spatial variable $\vec{r}$ will sweep out an incremental hexahedral volume $dV = dx \, dy \, dz$ about $\vec{r}$.

Similarly, an incremental differential surface on the unit sphere, $d \hat{\Omega}$, is swept out about $\hat{\Omega}$ by displacing the angular variables $\mu = \cos \theta$ and $\psi$ by $d\mu$ and $d \psi$. The resulting differential volume $d \hat{\Omega} = d \mu \, d \psi$, resides on the surface of the unit sphere and the resulting total surface area of the unit sphere is 
\begin{equation}
  \text{Area} = \int_{4 \pi} d \hat{\Omega} = \int_{0}^{2\pi} \int_{-1}^1 d\mu d\psi = 4 \pi.
\end{equation}

Finally, if we displace the energy variable, $E$, by an incremental amount $dE$ we can write the total six-dimensional phase space at time $t$ as
\begin{equation}
  dp = dV \, dE \, d\hat{\Omega}.
\end{equation}

\subsection{Angular particle density and angular flux}
We define the distribution function for a given particle species as $f(\vec{r},E,\hat{\Omega},t)$, that is, the probability that a particle is located at position $\vec{r}$ with energy $E$ traveling in direction $\hat{\Omega}$ at time $t$. Multiplying $f$ by the phase space, $dp$, gives the expected number of particles in the differential phase space for a given phase space at time $t$,
\begin{equation}
  N(\vec{r},E,\hat{\Omega},t) = f(\vec{r},E,\hat{\Omega},t) \, dp.
\end{equation}
The quantity $N(\vec{r},E,\hat{\Omega},t)$ is often referred to as the angular particle density and has units of $(\cm^{-3} \mev^{-1})$. The quantity of interest in this work is the angular flux and is related to the angular particle density by 
\begin{equation}
  \Psi(\vec{r},E,\hat{\Omega},t) = v \, N(\vec{r},E,\hat{\Omega},t)
\end{equation}
where $v$ is the particles velocity.
Therefore, the volume based interpretation of the angular flux, or $\Psi(\vec{r},E,\hat{\Omega},t) \,dV\,dE\,d\hat{\Omega}$, is the rate at which path length is generated by particles in $dV\,dE\,d\hat{\Omega}$ about $(\vec{r}, E, \hat{\Omega})$ at time $t$ with units of $(\cm^{-2} \mev^{-1} \text{s}^{-1})$. Many other derived quantities follow from the angular flux or the scalar flux which is given by
\begin{equation}
  \Phi(\vec{r},E,t) = \int_{4\pi} d \hat{\Omega} \,\, \Psi(\vec{r},E,\hat{\Omega},t),
\end{equation}
and is the zeroth angular moment of the angular flux.

\subsection{Cross section definitions}
Using the foregoing assumptions we can define the microscopic and macroscopic cross sections and relate them to the particle interaction probabilities. Consider a beam of particles streaming with direction $\hat{\Omega}$ and energy $E$ impinging on a slab of material with area $A$ and thickness $ds$ comprised of atoms of a single isotope. The microscopic cross section $\sigma(E)$ is the effective cross-sectional area per target nucleus seen by the incident particles. Microscopic cross sections are typically measured in square centimeters or barns $(1 \text{barn} = 10^{-24} \cm^2)$. Suppose that the number of nuclei per unit volume in the slab is $\mathcal{N}$. 

\subsection{Limitations of the transport equation}
The Boltzmann transport equation provides an accurate description of the angular particle density across the domain of the problem. However, the derivation of the transport equation has some inherent approximations that limit the validity of the solution. The validity of these approximations are now examined and applied to our energetic ion problems.
\begin{enumerate}
  \item \textit{Particles are treated as points, and whose free motion between collisions can be described using classical mechanics.} The energetic ions properties that require a quantum mechanical description are assumed to only influence the differential cross sections that describe collisions between particles.
  
  \item \textit{The interaction centers, where particles collide with background, are assumed to be randomly located within a Poisson distribution in the host medium, and collisions between the incident and target particles are
  well-defined isolated interactions. Moreover, the probability of interaction with the
  target particle is assumed to depend only on the instantaneous state of the incident particle and not on its
  prior interaction history.} More generally the transport process is assumed to be Markovian, that is, the probability of interaction in an incremental distance $ds$ is proportional to $ds$ and depends only on the ions current position and energy at the time of collision.

  \item \textit{The transport equation, as introduced in the next section, describes the “expected” or “mean”
  particle population in a medium.} Due to the statistical nature of the Boltzmann transport equation fluctuations about the mean due to low particle density density are not considered. When low-density situations are encountered it is assumed that the density distribution can be integrated over a large enough phase-space or time span such that the mean value has meaning.

  \item \textit{The transport equation is a linear equation in which the macroscopic cross sections
  do not directly depend on the changing density of the background medium.} Two further assumptions are required to ensure this independence:
  \begin{enumerate}
    \item \textit{Energetic ion-ion collisions are negligible.} To that end we assume that two ions with energies above the thermal background energies cannot interact with each other.
    \item \textit{The ion reaction rate density is sufficiently small so as to not appreciably modify the
    isotopic makeup of the host medium, except during very long timescales.} In other words we assume that the background is not changing appreciably quickly such that the background density of ions is not significantly changing.
  \end{enumerate}
\end{enumerate}