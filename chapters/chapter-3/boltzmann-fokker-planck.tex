\section{Boltzmann-Fokker-Planck}
To account for the large-angle scattering collisions that the FP scattering operator neglects, the Boltzmann-Fokker-Planck (BFP) scattering operator is introduced \cite{Ligou-1986}. The BFP scattering operator splits the differential cross section into two parts: one part that is peaked and one part that is smooth, that is,
\begin{equation}
    \sigma_{x,t \rightarrow x}(E^{\prime} \rightarrow E; \mucm) = \sigma_{x,t \rightarrow x}^{p}(E^{\prime} \rightarrow E; \mucm) + \sigma_{x,t \rightarrow x}^{s}(E^{\prime} \rightarrow E; \mucm),
\end{equation}
where $\sigma_s^{p}$ and $\sigma_s^{s}$ are the peaked and smooth parts respectively. The peaked part is then represented by an FP scattering operator, while the smooth part is represented by a Boltzmann scattering operator. The BFP scattering operator is expressed as
\begin{equation} \label{eqn:BFP-operator}
    \boldsymbol{C}_{x,t \rightarrow x}^{BFP} = \boldsymbol{C}_{x,t \rightarrow x}^{B} + \boldsymbol{C}_{x,t \rightarrow x}^{FP},
\end{equation}
where the superscripts $B$ and $FP$ represent the Boltzmann and FP scattering operators respectively.

To ensure that the BFP scattering operator is an accurate approximation to the original Boltzmann scattering operator, the splitting of the original differential cross section needs to be done in such a way that the peaked differential cross section satisfies the asymptotic limit of the FP expansion, and the MFP associated with new Boltzmann scattering operator is $\mathcal{O}(1)$.

\subsection{Decomposition methods}