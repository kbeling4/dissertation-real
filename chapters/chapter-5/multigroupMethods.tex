\section{Multigroup data}
In this section the algorithms used to compute the multigroup charged particle data are introduced. First we present the method used to compute the group constants, that is, the multigroup form of cross sections that are only energy dependent. Next we present the algorithms used to compute the transfer matrices.

\subsection{Group Constants}
The group constants are computed in the usual manner
\begin{equation}
  \sigma_g = \dfrac{\int_{I_g} \sigma(E) \, w(E) \, dE}{\int_{I_g} w(E) \, dE}
\end{equation}
where $w(E)$ is some weighting function.

\subsection{Transfer Matrices}
Consider the definition of the group to group transfer
\begin{equation} \label{eqn:g2g_transfer}
  \int\limits_{E_{g-\frac{1}{2}}}^{E_{g+\frac{1}{2}}} dE \int\limits_{4 \, \pi} d\hat{\Omega}^{\prime} \int\limits_{E_{g^{\prime}-\frac{1}{2}}}^{E_{g^{\prime}+\frac{1}{2}}} dE^{\prime} \,\, \Sigma(E^{\prime}\rightarrow E; \hat{\Omega}^{\prime} \rightarrow \hat{\Omega}) \,\, \Psi(E^{\prime},\hat{\Omega}^{\prime})
\end{equation}
By assuming that the background is homogeneous and isotropic, the scattering becomes a function of the angle between the initial and final directions, $\hat{\Omega}^{\prime}$ and $\hat{\Omega}$. The cosine of the scattering angle, $\mucm$ , can be written in terms of the initial and final scattering angles as
\begin{equation}
 \mucm = \hat{\Omega}^{\prime} \cdot \hat{\Omega} = \mu^{\prime} \mu + \sqrt{1-\left(\mu^{\prime}\right)^2} \sqrt{1-\mu^{2}} \cos\left(\phi^{\prime} - \phi \right) 
\end{equation}
and in general
\begin{equation} \label{eqn:mucm_dxs}
  \Sigma(E^{\prime}\rightarrow E; \hat{\Omega}^{\prime} \rightarrow \hat{\Omega}) = \Sigma(E^{\prime}\rightarrow E; \mucm).
\end{equation}

To convert this into a function of initial and final directions, Eq. \eqref{eqn:mucm_dxs} is expanded in a series of legendre polynomials
\begin{equation} \label{eqn:legendre_expansion_dxs}
  \Sigma(E^{\prime}\rightarrow E; \mucm) = \sum_{l=0}^L \dfrac{2l+1}{2} \, \Sigma_l(E^{\prime}\rightarrow E) \, P_l(\mucm)
\end{equation}
where
\begin{equation}
  \Sigma_l(E^{\prime}\rightarrow E) = \int\limits_{-1}^{1} d\mucm \, P_l(\mu) \, \Sigma(E^{\prime}\rightarrow E; \mucm).
\end{equation}
Substituting Eq. \eqref{eqn:legendre_expansion_dxs} back into Eq. \eqref{eqn:g2g_transfer} and using the relationship,
\begin{equation}
  P_l(\mucm) = \sum\limits_{m=-l}^l Y_{n}^{m}(\hat{\Omega}^{\prime}) \, Y_{n}^{m}(\hat{\Omega}),
\end{equation}
gives
\begin{equation}
  \sum_{l=0}^{\infty} \dfrac{2l+1}{2} \sum\limits_{m=-l}^l Y_{n}^{m}(\hat{\Omega}) \int\limits_{E_{g-\frac{1}{2}}}^{E_{g+\frac{1}{2}}} dE \int\limits_{4 \, \pi} d\hat{\Omega}^{\prime} \, Y_{n}^{m}(\hat{\Omega}^{\prime}) \int\limits_{E_{g^{\prime}-\frac{1}{2}}}^{E_{g^{\prime}+\frac{1}{2}}} dE^{\prime} \, \Sigma_l(E^{\prime}\rightarrow E) \, \Psi(E^{\prime},\hat{\Omega}^{\prime}).
\end{equation}
Next, defining the spherical harmonic moment of the flux as
\begin{equation}
  \Phi_l^m(E^{\prime}) = \int_{4\,\pi} d\hat{\Omega}^{\prime} \,\, Y_{n}^{m}(\hat{\Omega}^{\prime}) \,\, \Psi(E^{\prime},\hat{\Omega}^{\prime}),
\end{equation}
gives the final form of the expression as
\begin{equation} \label{eqn:transfer-matrix-full}
  \sum_{l=0}^{\infty} \dfrac{2l+1}{2} \sum\limits_{m=-l}^l Y_{n}^{m}(\hat{\Omega}) \int\limits_{E_{g-\frac{1}{2}}}^{E_{g+\frac{1}{2}}} dE \int\limits_{E_{g^{\prime}-\frac{1}{2}}}^{E_{g^{\prime}+\frac{1}{2}}} dE^{\prime} \, \Sigma_l(E^{\prime}\rightarrow E) \, \Phi_l^m(E^{\prime}).
\end{equation}
which expresses the transfer to the final direction $\hat{\Omega}$, in terms of the spherical harmonics moments of the flux. Finally, in the azimuthally symmetric system Eq. \eqref{eqn:transfer-matrix-full} can be simplified by integrating over the azimuthal angle; in this case, all the terms will be zero except those with $m = 0$, giving
\begin{equation} \label{eqn:transfer-matrix-simple}
  \sum_{l=0}^{\infty} \dfrac{2l+1}{2}  P_{l}(\mu) \int\limits_{E_{g-\frac{1}{2}}}^{E_{g+\frac{1}{2}}} dE \int\limits_{E_{g^{\prime}-\frac{1}{2}}}^{E_{g^{\prime}+\frac{1}{2}}} dE^{\prime} \, \Sigma_l(E^{\prime}\rightarrow E) \, \Phi_l(E^{\prime})
\end{equation}
where
\begin{equation}
  \Phi_l(E) = \int\limits_{-1}^{1} d\mu \, P_l(\mu) \, \Psi(E,\mu).
\end{equation}
From Eq. \eqref{eqn:transfer-matrix-simple} the group constants are defined as
\begin{equation} \label{eqn:group-constants}
  \sigma_{g^{\prime}\rightarrow g,l} = \int\limits_{E_{g-\frac{1}{2}}}^{E_{g+\frac{1}{2}}} dE \int\limits_{E_{g^{\prime}-\frac{1}{2}}}^{E_{g^{\prime}+\frac{1}{2}}} dE^{\prime} \Phi_l(E^{\prime}) \int\limits_{-1}^{1} d\mucm \, P_l(\mu) \, \Sigma(E^{\prime}\rightarrow E; \mucm).
\end{equation}

In the case of correlated distributions, the secondary distribution is expressed as the product of the angular distribution and a Dirac delta function expressing the exact correlation between scattering angle and initial and final energies, as
\begin{equation} \label{eqn:energy-angle-expression}
  \Sigma(E^{\prime},E,\mucm) = \Sigma(E^{\prime},\mucm) \, \delta\left[\mucm - S(E^{\prime},E)\right]
\end{equation}
where $S(E^{\prime},E)$ describes the relationship between the scattering angle and the incoming and outgoing energies. Substituting Eq. \eqref{eqn:energy-angle-expression} back into Eq. \eqref{eqn:group-constants} gives the final expression for computing the terms in the transfer matrix which can be written in the laboratory system as
\begin{equation} \label{eqn:transfer-constants}
  \sigma_{g^{\prime}\rightarrow g,l} = \int\limits_{E_{g-\frac{1}{2}}}^{E_{g+\frac{1}{2}}} dE \int\limits_{E_{g^{\prime}-\frac{1}{2}}}^{E_{g^{\prime}+\frac{1}{2}}} dE^{\prime} \, w(E^{\prime}) \int\limits_{-1}^{1} d\mu \, P_l(\mu) \, \Sigma(E^{\prime},\mucm[\mu]) \, \left| \dfrac{d\mucm}{d\mu} \right| \, \delta\left[\mu - S(E^{\prime},E)\right].
\end{equation}

Because of the delta function in Eq. \eqref{eqn:transfer-constants} we can combine the angular integral and integral over the outgoing energy group into a single integral over the angular variable with integration limits as
\begin{equation}
  \sigma_{g^{\prime}\rightarrow g,l} = \int\limits_{E_{g^{\prime}-\frac{1}{2}}}^{E_{g^{\prime}+\frac{1}{2}}} dE^{\prime} \, w(E^{\prime}) \int\limits_{\omega_{\min}[E^{\prime}]}^{\omega_{\max}[E^{\prime}]} d\mu \, P_l(\mu) \, \Sigma(E^{\prime},\mucm[\mu]) \, \left| \dfrac{d\mucm}{d\mu} \right| \, \delta\left[\mu - S(E^{\prime},E)\right]
\end{equation}
where $\omega_{\min}$ and $\omega_{\max}$ are the following functions of $E^{\prime}$:
\begin{subequations} \label{eqn:integration-limits-transfer-matrices}
  \begin{align}
    \omega_{\min}[E^{\prime}] &= \min\left\lbrace \mu_{\max}, \max\left[\mu_{\min}, S(E^{\prime}, E_{g-\frac{1}{2}})\right]\right\rbrace \\
    \omega_{\max}[E^{\prime}] &= \max\left\lbrace \mu_{\min}, \min\left[\mu_{\max}, S(E^{\prime}, E_{g+\frac{1}{2}})\right]\right\rbrace
  \end{align}
\end{subequations}
In Eq. \eqref{eqn:integration-limits-transfer-matrices}, the energy-angle relationships $S(E^{\prime},E)$ are given in Section 1 of this chapter and the minimum and maximum cosine of the scattering angle are the corresponding minimum and maximum cosine angles for the relationship being used. 