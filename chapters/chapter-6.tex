\documentclass[../main.tex]{subfiles}

\begin{document}

\chapter{Numerical Discretizations}
In this section, the numerical methods used to discretize the various operators in the CPT equation are introduced, and several numerical experiments are performed to demonstrate that these operators have been discretized properly. To simplify notation, the CPT equation describing a single ion is written in operator notation as
\begin{equation} \label{eqn:operator-equation}
    \boldsymbol{L} \vec{\Psi}_x + \boldsymbol{P} \vec{\Psi}_x = \boldsymbol{C}_{x} \vec{\Psi}_x + \vec{Q}_x,
\end{equation}
where
\begin{subequations}
    \begin{equation}
        \boldsymbol{L} \vec{\Psi}_x = \dfrac{\mu}{r^2} \dfrac{\partial}{\partial r}\Big[ r^2 \Psi_x(r,E,\mu)\Big] + \Sigma_{a,x}(r,E) \Psi_x(r,E,\mu),
    \end{equation}
    and
    \begin{equation}
        \boldsymbol{P} \vec{\Psi}_x = \dfrac{1}{r} \dfrac{\partial}{\partial \mu}\Big[ (1-\mu^2) \Psi_x(r,E,\mu)\Big].
    \end{equation}
\end{subequations}
The operators on the left-hand side of Eq.~\eqref{eqn:operator-equation} are the spatial and angular streaming operators, and the scattering collision operator, $\boldsymbol{C}_{x}$, is either a Boltzmann collision operator, an FP collision operator, or some combination of these two collision operators such as the BFP collision operator. In this section only the Boltzmann and FP collision operators are discretized because all other possible CPT collision operators are combinations of these two collision operators.

In this work, the angular domain is discretized using the discrete ordinates (\Snns) method, and the spatial and energy domains are discretized using a discontinuous Galerkin (\dGns) finite element method. This section introduces the discretization's of each domain separately, as the fully discretized form of any CPT operator is calculated by using Kronecker products to multiply each discretized domain together. For example if $\boldsymbol{L}^A$, $\boldsymbol{L}^S$, and $\boldsymbol{L}^E$ are the angular, spatial, and energy discretized spatial streaming operators, then the fully discretized spatial streaming operator is given by
\begin{equation}
    \boldsymbol{L}^D = \boldsymbol{L}^A \otimes \boldsymbol{L}^S \otimes \boldsymbol{L}^E.
\end{equation}

The remainder of this section is divided into four parts: 1) \Sn angular discretization method, 2) \dG spatial discretization method, 3) \dG energy discretization method, and 4) numerical experiments. The first three parts present the various methods used to discretize the operators in Eq.~\eqref{eqn:operator-equation} on each domain of the problem, while the fourth part uses the method of manufactured solutions (MMS) to verify that the various operators have been discretized properly over the entire domain of the problem.

% Sn ANGULAR DISCRETIZATION
\section{\Sn angular discretization}
The method used to discretize the angular domain in this work is the discrete ordinates method, which consists of evaluating the CPT equation at discrete angles using a set of discrete angular directions. Here a one-dimensional \Sn approximation is employed by using a set of $N$ directions $(\mu_m)$ and weights $(w_m)$. These directions and weights are chosen such that they satisfy the N-point quadrature rule, that is,
\begin{equation}
    \int_{a}^b f(x) \, dx \approx w_1 \, f(\mu_1) + w_2 \, f(\mu_2) + \ldots + w_N \, f(\mu_N) = \sum_{m=1}^N w_{m} \, f(\mu_m).
\end{equation}
In this work only Gauss-type quadratures are considered such as Gauss-Legendre and Gauss-Lobatto. Gauss-Lobatto quadrature is especially useful in spherical geometries because it includes the points of the angular domain, $+1$ and $-1$, where the angular streaming term vanishes. 

Using the \Sn method, a vector of $N$ discrete angular fluxes corresponding to the directions $\mu_m$ is written as $\vec{\Psi}_x(r,E)$, where an element $m$ of the vector corresponds to the discrete angular flux $\Psi_x(r,E,\mu_m)$. The angular discretization is trivial for all operators in Eq.~\eqref{eqn:operator-equation} except for the angular streaming operator, Boltzmann collision operator, and the angular diffusion operator in the FP collision operator as they operate in angle on the angular flux. The angular integrations required to calculate the angular moments in the Boltzmann collision operator are computed using the quadrature rule, that is,
\begin{equation}
    \Psi_l(r,E) = \int_{-1}^1 \, P_l(\mu) \, \Psi(r,E,\mu) \, d\mu = \sum_{m=0}^N w_m \, P_l(\mu_m) \, \Psi_m(r,E).
\end{equation} 

In remainder of this section, the weight-diamond-difference (WDD) and differential quadrature (DQ) angular discretization methods for the angular streaming and angular diffusion operators are discussed. 

\subsection{The weighted diamond difference method}
The WDD discretization of the angular streaming and angular FP collision operators at some discrete angular direction, $\mu_m$, are
\begin{subequations} \label{eqn:WDD_angular_operators}
    \begin{align}
        \dfrac{\partial}{\partial \mu} \left.\left[(1-\mu^2) \Psi(r,E,\mu) \right]\right|_{\mu_m} &= \dfrac{1}{w_m} \left( \alpha_{m+1/2} \Psi_{m+1/2}(r,E) - \alpha_{m-1/2} \Psi_{m-1/2}(r,E) \right), \\
        \dfrac{\partial}{\partial \mu} \left.\left[(1-\mu^2) \frac{\partial \Psi}{\partial \mu} \right]\right|_{\mu_m} &= \dfrac{1}{w_m} \left( \beta_{m-1} \Psi_{m-1}(r,E) - \beta_{m+1} \Psi_{m+1}(r,E) \right),
    \end{align}
\end{subequations} 
where the half indices refer to the angular flux evaluated at the boundary of the angular cells. The WDD method closes the equations using the diamond-difference relationships
\begin{subequations}
    \begin{align}
        \Psi_m(r,E) &= (1 - \tau_m) \Psi_{m-1/2}(r,E) + \tau_m \Psi_{m+1/2}(r,E), \\
        \Psi_m(r,E) &= (1 - \gamma_m) \Psi_{m-1}(r,E) + \gamma_m \Psi_{m+1}(r,E),
    \end{align}
\end{subequations}
where the WDD constants for the angular streaming and angular FP collision operators are given here \cite{Morel-1984} \cite{Morel-1985}.

To solve the WDD relationships in Eqs.~\eqref{eqn:WDD_angular_operators} for the angular streaming operator, first a starting-direction equation with $\mu_{1/2} = -1$ must be solved, and then each subsequent angular equation can be solved for by stepping through the discrete angles. Therefore, the angular streaming operator is lower triangular and can be easily inverted by using the standard \Sn sweep method. This is a major advantage over the DQ method shown in the next section which as will be seen, is a dense matrix that cannot be easily inverted. Additionally, the angular FP operator is tri-diagonal and can be inverted using the standard source iteration algorithm by lagging the off diagonal terms \cite{Morel-1985}. The main disadvantage to the WDD difference method is that it only rigorously preserves the zeroth and first angular moments of the angular flux, whereas DQ methods have been shown to preserve a higher-order number of moments \cite{Warsa-2012} \cite{Warsa-2006}.

\subsection{The differential quadrature method}
Here the differential quadrature method of Bellman \cite{Bellman-1972} is introduced and used to discretize the angular streaming and angular FP collision operators. The DQ method makes the approximation that the derivative of a function $f(x)$ with respect to $x$ at a grid point $x_i$, is approximated by a linear sum of all the functional values in domain, that is,
\begin{equation}
    f_x(x_i) = \left.\dfrac{df}{dx}\right|_{x_i} = \sum_{j=1}^N a_{ij} \, f(x_j), \quad \text{for} \, i=1,2,\ldots,N
\end{equation} 
where $a_{ij}$ represent the weighting coefficients. The key procedure in the DQ approximation is to determine the coefficients $a_{ij}$, and this procedure is summarized here using a Lagrange interpolator basis. 

To derive a formula for the weighting coefficients $a_{ij}$, the Lagrange interpolation polynomials are
\begin{equation}
    g_k(x) = \dfrac{M(x)}{(x - x_k) M^{\prime}(x_k)}
\end{equation}
where the polynomials
\begin{equation}
    M(x) = \prod_{j=1}^N (x - x_j)
\end{equation}
and $M^{\prime}(x)$ is the first order derivative of $M(x)$ for some set of nodes $x_i, \,\, i = 1,\ldots,N$. Next, it is assumed that the function $f(x)$ can be represented using these interpolating polynomials as
\begin{equation}
    f(x) = \sum_{j=1}^N f_j \, g_j(x)
\end{equation}
such that $f(x_i) = f_i$. The derivative of $f(x)$ is
\begin{equation}
    f^{\prime}(x) = \sum_{j=1}^N f_j \, g_j^{\prime}(x).
\end{equation}
From this, the derivatives $f^{\prime}(x_i)$ can be determined by evaluating the expansion for the derivative at $x_i$ with another expansion
\begin{equation}
    f^{\prime}(x_i) = \sum_{j=1}^N f_j \, g_j^{\prime}(x_i) = \sum_{j=1}^N a_{ij} \, f_j
\end{equation}
whose coefficients $a_{ij}$ are to be determined. When $i \neq j$ the coefficients are given by
\begin{equation}
    a_{ij} = g_j^{\prime}(x_i) = \dfrac{M^{\prime}(x_i)}{(x_i - x_j) \, M^{\prime}(x_j)}
\end{equation}
where 
\begin{equation}
    M^{\prime}(x_i) = \prod\limits_{\substack{j = 1 \\ j \neq i}}^N (x_i - x_j),
\end{equation}
and when $i = j$ it can be shown \cite{Shu-2000} that coefficients are given by
\begin{equation}
    a_{ii} = \sum\limits_{\substack{j=1 \\ i \neq j}}^N - a_{ij}.
\end{equation}
Finally, computing the expansion coefficients gives a transformation matrix, $\boldsymbol{A}$, such that when multiplied against a vector $v$, whose elements are $v_i = v(x_i)$,
\begin{equation} \label{eqn:DQ_method}
    v^{\prime} = \boldsymbol{A} v
\end{equation}
gives a vector $v^{\prime}$ whose elements are the derivatives of $v(x)$ at the nodes $x_i$. 

Using the DQ method, both the angular streaming and AFP operators are represented using the vector notation above as some matrix, $\boldsymbol{R}_{op}$, that describes the angular derivative. Using Eq.~\eqref{eqn:DQ_method}, the angular streaming operator, $\boldsymbol{R}_{sph}$, is written as
\begin{equation} \label{eqn:angular_discretization_angular_streaming}
    \boldsymbol{R}_{sph} = \left[ (\boldsymbol{I} - \boldsymbol{M}^2) \boldsymbol{A} - 2 \boldsymbol{M} \right],
\end{equation}
and the AFP operator is
\begin{equation} \label{eqn:angular_discretization_afp}
    \boldsymbol{R}_{afp} = \left[ (\boldsymbol{I} - \boldsymbol{M}^2) \boldsymbol{A}^2 - 2 \boldsymbol{M} \boldsymbol{A} \right],
\end{equation}
where $\boldsymbol{M} = \text{diag} \, (\mu_m)$.

Advantages of the DQ method over the WDD method are that the DQ method possesses a quicker convergence to the solution since it relies on interpolator polynomials to approximate the original function. However, the result is a dense matrix that cannot be inverted efficiently like the WDD matrix can.

% dG SPATIAL DISCRETIZATION
\section{\dG spatial discretization}
The spatial discretizations of the CPT operators are done using a \dG finite element method. To begin, the finite spatial range $r \in [0,R]$ is divided into $K$ intervals order as
\begin{equation}
    0 = r_{\frac{1}{2}} < r_{\frac{3}{2}} < \ldots < r_{i-\frac{1}{2}} < r_{i+\frac{1}{2}} < \ldots < r_{K+\frac{1}{2}} = R,
\end{equation}
with a spatial cell $i$ being the interval $I_i = [r_{i-\frac{1}{2}}, r_{i+\frac{1}{2}}]$. Next, the finite element basis $P^p(I_i)$ of degree $p$ on interval $I_i$ consisting of the functions $\lbrace u_j(r) \in P^p(I_i), \, 1 \leq j \leq p \rbrace$ is introduced. The solution $\Psi_x(r,E,\mu)$ is then expanded in the local polynomial basis for cell $i$,
\begin{equation} \label{eqn:spatial-expansion}
    \Psi_{x,i}(r,E,\mu) = \sum_{j=1}^p \Psi_{x,i}^j(E,\mu) u_j(r).
\end{equation}
Substituting Eq.~\eqref{eqn:spatial-expansion} into the CPT operators, multiplying by the test functions $\lbrace v_k(r) \in P^p(I_i), \, 1 \leq k \leq p \rbrace$, and integrating over the spherical shell gives the weak forms of each CPT operator. It is noted that only the spatial streaming operator, $\boldsymbol{L}$, operates in space on the angular flux, and therefore all other operators discretized in space result in spatial mass matrices. These discretizations are not shown here because they are straightforward; instead only the discretization of the spatial streaming operator is shown.

The weak form of the spatial streaming operator is 
\begin{multline} \label{eqn:spatial-weak-form}
    \boldsymbol{L} \Psi_x = \mu \, v_k r_{i+1/2}^2 \widehat{\Psi}_{i+1/2} - \mu \, v_k r_{i-1/2}^2 \widehat{\Psi}_{i-1/2} - \mu \, \int_{I_i} dr \, r^2 \, v_k^{\prime} \Psi_{x,i}(r,E,\mu) \\ 
    + \Sigma_{a,z} \int_{I_i} dr \, r^2 \, v_k \, \Psi_{x,i}(r,E,\mu),
\end{multline}
where the terms $\widehat{\Psi}$ are the numerical fluxes that introduce discontinuities into the discretization via upwinding based on the ``flow direction'' of the particles. These ``flow directions'' are defined in the spatial streaming operator because particles clearly flow in the spatial direction associated with direction cosine $\mu$. For example, if $\mu > 0$ then the particles will travel from left to right across the spatial domain of the problem. Therefore, the numerical fluxes associated with the spatial streaming operator, $\widehat{\Psi}$, are defined as
\begin{subequations} \label{eqn:spatial-upwinding}
    \begin{align}
        \widehat{\Psi}_{i-1/2} &= \Psi_{z,i-1}(r_{i-\frac{1}{2}},E,\mu),  &\text{for} \, \mu > 0 \\
        \widehat{\Psi}_{i-1/2} &= \Psi_{z,i}(r_{i-\frac{1}{2}},E,\mu),  &\text{for} \, \mu < 0 \\
        \widehat{\Psi}_{i+1/2} &= \Psi_{z,i}(r_{i+\frac{1}{2}},E,\mu),  &\text{for} \, \mu > 0 \\
        \widehat{\Psi}_{i+1/2} &= \Psi_{z,i+1}(r_{i+\frac{1}{2}},E,\mu),  &\text{for} \, \mu < 0.
    \end{align}
\end{subequations}

% dG ENERGY DISCRETIZATION
\section{\dG energy discretization}
Similar to the spatial discretization, the energy discretization of the CPT operators in Eq.~\eqref{eqn:operator-equation} are done using a \dG finite element method. This method begins by breaking the energy domain, $E \in [E_{min}, E_{max}]$, into $G$ groups defined as
\begin{equation}
    E_{min} = E_{\frac{1}{2}} < E_{\frac{3}{2}} < \ldots < E_{g-\frac{1}{2}} < E_{g+\frac{1}{2}} < \ldots < E_{G+\frac{1}{2}} = E_{max},
\end{equation}
with an energy group $g$ being defined as $I_g = [E_{g-\frac{1}{2}}, E_{g+\frac{1}{2}}]$. Next, the finite element basis $P^p(I_g)$ of degree $p$ on interval $I_g$ consisting of the functions $\lbrace u_j(E) \in P^p(I_g), \, 1 \leq j \leq p \rbrace$ is defined. The solution $\Psi_x(r,E,\mu)$ is expanded in the local polynomial basis for an energy group $g$ as
\begin{equation} \label{eqn:energy-expansion}
    \Psi_{x,g}(r,E,\mu) = \sum_{j=1}^p \Psi_{x,g}^j(r,\mu) u_j(E).
\end{equation}
Substituting Eq.~\eqref{eqn:energy-expansion} into the CPT operators, multiplying by the test functions $\lbrace v_k(E) \in P^p(I_g), \, 1 \leq k \leq p \rbrace$, and integrating over an energy group gives the weak forms for each operator. Note that only the Boltzmann and FP collision operators act in energy on the angular flux, and therefore the \dG discretizations of the remaining operators simply result in mass matrices. Only the \dG energy discretizations of the Boltzmann and FP collision operators are shown in the remainder of this section, because the discretization to a mass matrix is considered trivial.

\subsection{\dG energy discretization of the Boltzmann collision operator}
The dG(p) energy discretization of the Boltzmann collision operator, Eq.~\eqref{eqn:boltzmann_collision_operator_legendre}, starts by breaking up the integral over $E^{\prime}$ into a sum over energy groups as
\begin{multline} \label{eqn:boltzmann-ed1}
    \boldsymbol{C}_{x,t}^B \Psi_z = \sum_{l=0}^{\infty} \dfrac{2l+1}{4 \pi} \, P_l(\mu)  \sum_{g^{\prime}=1}^G \int_{I_{g^{\prime}}} 
    dE^{\prime} \, \Sigma_{x,t\rightarrow x, l}(E^{\prime} \rightarrow E) \, \Psi_{x,l}(r,E^{\prime}) \\
    - \Sigma_{x,t \rightarrow x}(r,E) \Psi_x(r,E,\mu).
\end{multline}
Next, substituting Eq.~\eqref{eqn:energy-expansion} into Eq.~\eqref{eqn:boltzmann-ed1} gives the weak form of the Boltzmann collision operator,
\begin{multline} \label{eqn:boltzmann-ed2}
    \boldsymbol{C}_{x,t}^B \Psi_x = \sum_{l=0}^{\infty} \dfrac{2l+1}{4 \pi} \, P_l(\mu) \int_{I_g} dE \, v_k  \sum_{g^{\prime}=1}^G \int_{I_{g^{\prime}}} 
    dE^{\prime} \, \Sigma_{x,t\rightarrow x, l}(E^{\prime} \rightarrow E) \, \Psi_{x,l,g^{\prime}}(r,E^{\prime}) \\
    - \int_{I_g} dE \, v_k \Sigma_{x,t\rightarrow x}(r,E) \Psi_x(r,E,\hat{\Omega}).
\end{multline}
To proceed, both the total and differential scattering cross sections are assumed to be appropriately averaged in energy, that is,
\begin{subequations} \label{eqn:group-averaged-xs}
    \begin{equation} 
        \Sigma_{x,t \rightarrow x}(r,E) = \Sigma_{x,t \rightarrow x, g}(r),
    \end{equation}
    \begin{equation}
        \Sigma_{x,t \rightarrow x,l}(E^{\prime} \rightarrow E) = \Sigma_{x,t\rightarrow x,l,g^{\prime}\rightarrow g}(r).
    \end{equation}
\end{subequations}
Finally, substituting Eq.~\eqref{eqn:group-averaged-xs} into Eq.~\eqref{eqn:boltzmann-ed2} the \dG discretized form of the Boltzmann collision operator is
\begin{multline} \label{eqn:boltzmann-ed3}
    \boldsymbol{C}_{x,t}^B \Psi_x = \sum_{l=0}^{\infty} \dfrac{2l+1}{4 \pi} \, P_l(\mu) \int_{I_g} dE \, v_k  \sum_{g^{\prime}=1}^G \Sigma_{x,t\rightarrow x,l,g^{\prime}\rightarrow g}(r) \int_{I_{g^{\prime}}} 
    dE^{\prime} \, \Psi_{x,l,g^{\prime}}(r,E^{\prime}) \\
    - \Sigma_{x,t\rightarrow x,g}(r) \, \int_{I_g} dE \, v_k \, \Psi_{x,g}(r,E,\mu).
\end{multline}

\subsection{\dG energy discretization of the Fokker-Planck collision operator}
Here the \dG energy discretization of the FP collision operator is presented. Note that only a portion of the total FP collision operator operates in energy on the angular flux. Only the portion that operates in energy on the angular flux is discretized here, because the \dG energy discretization  of the the angular diffusion operator results in an energy mass matrix. The energy-dependent portion of the FP collision operator is,
\begin{equation} \label{eqn:FP_energy_dependent}
    \boldsymbol{C}_{x,t}^{FPE} \Psi_x = \dfrac{\partial}{\partial E} \left\lbrace S_{x,t\rightarrow x}(r,E) \Psi_z(r,E,\mu)  + \dfrac{1}{2} \dfrac{\partial}{\partial E} \Big[ T_{x,t\rightarrow x}(r,E) \Psi(r,E,\mu) \Big] \right\rbrace.
\end{equation}

Before proceeding to the \dG energy discretization, recall that Eq.~\eqref{eqn:FP_energy_dependent} is a second-order differential equation in energy and therefore requires two boundary conditions (BCs) in energy. In this work the zero-incoming energy-current BCs are used. The zero-incoming energy-current BCs are found by rewriting Eq.~\eqref{eqn:FP_energy_dependent} as an energy current $J(r,E,\mu)$, that is,
\begin{equation}
    \boldsymbol{C}_{x,t}^{FPE} \Psi_x = \dfrac{\partial J}{\partial E} = \dfrac{\partial}{\partial E} \left\lbrace S_{x,t\rightarrow x}(r,E) \Psi_x(r,E,\mu)  + \dfrac{1}{2} \dfrac{\partial}{\partial E} \Big[ T_{x,t\rightarrow x}(r,E) \Psi_x(r,E,\mu) \Big] \right\rbrace.
\end{equation}
Before proceeding, the FP moments of the DCS are assumed to be the following appropriately averaged multigroup constants
\begin{equation} \label{eqn:multigroup_FP_moments}
    S_{x,t\rightarrow x}(r,E) = S_g(r), \quad T_{x,t\rightarrow x}(r,E) = T_g(r).
\end{equation}
Using Eq.~\eqref{eqn:multigroup_FP_moments}, the energy current $J(r,E,\mu)$ is decomposed into group-wise partial currents using the method characteristic variables \cite{Warsa-2002}. The details of this transformation to characteristic variables are shown here \cite{Beling-2021}. The resulting partial currents are
\begin{subequations}
    \begin{align}
        J^+(r,E,\mu) &= -S_g(r) \, \gamma^+(r,E,\mu) + \sqrt{\frac{T_g(r)}{2}} \gamma^+(r,E,\mu), \\
        J^-(r,E,\mu) &=  \,\,\,\,S_g(r) \, \gamma^-(r,E,\mu) + \sqrt{\frac{T_g(r)}{2}} \gamma^-(r,E,\mu),
    \end{align}
\end{subequations}
where $\gamma^{\pm}(r,E,\mu)$ are ``directional flows'' in energy defined as
\begin{equation}
    \gamma^{\pm}(r,E,\mu) = \mp \Psi(r,E,\mu) + \dfrac{1}{2} \, \sqrt{\frac{T_g(r)}{2}} \dfrac{\partial \Psi}{\partial E}.
\end{equation}
Using these partial energy currents, the zero-incoming energy current boundary conditions are
\begin{subequations} \label{eqn:Fokker-Planck-BCS}
    \begin{align}
        J^+(r,E_{min},\mu) &= 0 = - \Psi(r,E_{min},\mu) + \sqrt{\dfrac{T_{0}(r)}{2}} \left.\dfrac{d \Psi}{dE}\right|_{E=E_{min}}, \\
        J^-(r,E_{max},\mu) &= 0 = \,\,\,\, \Psi(r,E_{max},\mu) + \sqrt{\dfrac{T_G(r)}{2}} \left.\dfrac{d \Psi}{dE}\right|_{E=E_{max}}.
    \end{align}
\end{subequations}
Finally, these zero-incoming energy current BCs allow the energy-dependent FP collision operator to be properly discretized in energy using a \dG method. 

To begin the \dG energy discretization, the FP collision operator is rewritten as the following coupled system of equations
\begin{subequations} \label{eqn:Fokker-Planck-de2}
    \begin{equation}
        S_g(r)  \dfrac{\partial}{\partial E} \Big[ \Psi_x(r,E,\mu) \Big] + \sqrt{\frac{T_g(r)}{2}} \dfrac{\partial}{\partial E} \Big[ R_x(r,E,\mu) \Big] = 0,
    \end{equation}
    \begin{equation}
        R_x(r,E,\mu) - \sqrt{\frac{T_g(r)}{2}} \dfrac{\partial}{\partial E} \Big[ \Psi_x(r,E,\mu) \Big] = 0.
    \end{equation}
\end{subequations}
In the exact same manner as $\Psi_x(r,E,\mu)$, the new variable $R_x(r,E,\mu)$ is expanded in the local polynomial basis for group $g$ as
\begin{equation} \label{eqn:R-energy-expansion}
    R_{x,g}(r,E,\mu) = \sum_{j=1}^k R_{x,g}^j(r,\mu) u_j(E).
\end{equation}
The weak form of Eq.~\eqref{eqn:Fokker-Planck-de2} is found by substituting Eqs.~\eqref{eqn:energy-expansion} and~\eqref{eqn:R-energy-expansion} into Eq.~\eqref{eqn:Fokker-Planck-de2}, multiplying by the test function $v_k$, and integrating over an energy group $g$. The weak form of the FP collision operator is
 \begin{subequations} \label{eqn:Fokker-Planck-weak-form}
    \begin{multline}
        v_k \, \tilde{\Psi}_{g+1/2} - v_k \, \tilde{\Psi}_{g-1/2} - S_g \int_{I_g} dE \, v_k^{\prime} \, \Psi_{x,g}(r,E,\mu) \\
        + v_k \, \hat{R}_{g+1/2} - v_k \, \hat{R}_{g-1/2} - \sqrt{\frac{T_g}{2}} \int_{I_g} dE \, v_k^{\prime} \, R_{x,g}(E) = 0,
    \end{multline}
    \begin{equation}
        \int_{I_g} dE \, v_k \, R_{x,g}(E) - v_k \, \hat{\Psi}_{g+1/2} + v_k \, \hat{\Psi}_{g-1/2} + \sqrt{\frac{T_g}{2}} \int_{I_g} dE \, v_k^{\prime} \, \Psi_{x,g}(E) = 0,
    \end{equation}
\end{subequations} 
where the terms $\tilde{\Psi}$, $\hat{\Psi}$, and $\hat{R}$ are the numerical fluxes that introduce discontinuities into the discretization via upwinding based on the flow direction of particles. Notice that these numerical fluxes contain the terms $S_g(r)$ and $\sqrt{T_g(r)/2}$ that will be assigned when the numerical fluxes are defined. 

The direction of flow is clearly defined for $\tilde{\Psi}$, which arises from the CSD operator, as particles can only lose energy in slowing down \cite{Lazo-1986}. As such, the numerical flux $\tilde{\Psi}$ is defined by the negatively directed flow in energy, that is,
\begin{equation}
    \tilde{\Psi}_{g+1/2} = S_{g+1} \, \Psi_{g+1}(E_{g+1/2}), \quad \tilde{\Psi}_{g-1/2} = S_g \, \Psi_{g}(E_{g-1/2}).
\end{equation}
For the energy-straggling operator, there is no clear definition of the flow directions $\hat{\Psi}$ and $\hat{R}$ since particles can both lose and gain energy as a result of this operator. To overcome this, two distinct methods for determining the upwinding are presented: the alternating flux \cite{Cheng-2008}, and the directional flows methods. 

In the alternating flux method, the direction of flows associated with the numerical fluxes $\hat{\Psi}$ and $\hat{R}$ are chosen arbitrarily to be fixed in the positive direction for one and in the negative direction for the other. The numerical flux $\hat{\Psi}$ is in this case chosen to flow in the negative direction to maintain consistency with $\tilde{\Psi}$. Thus, $\hat{R}$ is taken to flow in the positive direction, and the corresponding numerical fluxes are
\begin{subequations}
    \begin{equation}
        \hat{\Psi}_{g+1/2} = \sqrt{\frac{T_{g+1}}{2}} \, \Psi_{g+1}(E_{g+1/2}), \quad \hat{R}_{g+1/2} = \sqrt{\frac{T_{g}}{2}} \, R_{g}(E_{g+1/2}),
    \end{equation}
    \begin{equation}
        \hat{\Psi}_{g-1/2} = \sqrt{\frac{T_g}{2}} \, \Psi_{g}(E_{g-1/2}), \quad \hat{R}_{g-1/2} = \sqrt{\frac{T_{g-1}}{2}} \, R_{g-1}(E_{g-1/2}).
    \end{equation}
\end{subequations}

The directional flows method evaluates $\Psi(E)$ and $R(E)$ using the characteristic variables such that the numerical fluxes are defined as
\begin{subequations}
    \begin{equation}
        \hat{\Psi}_{g+1/2} = \gamma_{g+1/2}^- - \gamma_{g+1/2}^+, \quad \hat{R}_{g+1/2} = \gamma_{g+1/2}^- + \gamma_{g+1/2}^+
    \end{equation}
    \begin{equation}
        \hat{\Psi}_{g-1/2} = \gamma_{g-1/2}^- - \gamma_{g-1/2}^+, \quad \hat{R}_{g-1/2} = \gamma_{g-1/2}^- + \gamma_{g-1/2}^+
    \end{equation}
\end{subequations}
where the incoming flows into group $g$ are
\begin{subequations}
    \begin{align}
        \gamma_{g+1/2}^- &= \sqrt{\frac{T_{g+1}}{2}} \left[\,\,\,\,\frac{1}{2} \, \Psi_{g+1}(E_{g+1/2}) + \frac{1}{2} \, R_{g+1}(E_{g+1/2}) \right], \\
        \gamma_{g-1/2}^+ &= \sqrt{\frac{T_{g-1}}{2}} \left[-\frac{1}{2} \, \Psi_{g-1}(E_{g-1/2}) + \frac{1}{2} \, R_{g-1}(E_{g-1/2}) \right],
    \end{align}
\end{subequations}
and the outgoing flows from group $g$ are
\begin{subequations}
    \begin{align}
        \gamma_{g+1/2}^+ &= \sqrt{\frac{T_g}{2}} \left[-\frac{1}{2} \, \Psi_{g}(E_{g+1/2}) + \frac{1}{2} \, R_{g}(E_{g+1/2}) \right], \\
        \gamma_{g-1/2}^- &= \sqrt{\frac{T_g}{2}} \left[\,\,\,\,\frac{1}{2} \, \Psi_{g}(E_{g-1/2}) + \frac{1}{2} \, R_{g}(E_{g-1/2}) \right].
    \end{align}
\end{subequations}

\end{document}
