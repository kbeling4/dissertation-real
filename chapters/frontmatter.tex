\title{Finite Element Methods for Deterministic
    \\ Light Ion Transport in ICF Applications}

\author{Kyle S. Beling}

\degreesubject{M.S., Mathematics}

\degree{Doctor of Philosophy \\ Nuclear Engineering}

\documenttype{Dissertation}

\previousdegrees{B.S., Nuclear Engineering, University of New Mexico, 2018 \\
                  M.S., Nuclear Engineering, University of New Mexico, 2020}

\date{December, \thisyear}

\maketitle

\begin{dedication}
  To my parents, Albert II and Gladys, for their support,
  encouragement and the Corvette they're giving me for graduation. \\[3ex]
  ``A bird in hand is worth two in the bush''
        -- Anonymous
\end{dedication}

\begin{acknowledgments}
    \vspace{1.1in}
    I would like to thank my advisor, Professor Martin Sheen, for his support
    and some great action movies.  I would also like to thank my dog, Spot,
    who only ate my homework two or three times.  I have several other people
    I would like to thank, as well.\footnote{To my brother and sister, who
    are really cool.}
\end{acknowledgments}

\maketitleabstract %(required even though there's no abstract title anymore)

\begin{abstract}
    The theory of relativity is a real ``toughie'' to prove, but with the
    help of my family and my great grandpa Al, this paper presents the
    proof in its entirety.  Most of the math is correct, and the
    part about ``warp speed'' and ``parallel universe'' sounds very high-tech.
\clearpage %(required for 1-page abstract)
\end{abstract}

\tableofcontents
\listoffigures
\listoftables

\begin{glossary}{Longest  string}
    \item[CM] Center of mass frame of reference
    \item[\dG] discontinuous Galerkin discretization method
    \item[LAB] Laboratory frame of reference
\end{glossary}