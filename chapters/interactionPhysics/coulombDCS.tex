\section{Coulomb Differential Cross Sections}
The Coulomb differential cross section describes the collisions that charged particles experience in a medium as a result of interactions with the charges of the background charged particles present in the plasma. The unscreened and screened Coulomb differential cross sections are found by solving the Schrodinger equation, that is solutions to,
\begin{equation} \label{eqn:Schrodinger}
    \left[-\dfrac{\hbar^2}{2 m}\nabla^2 + k^2 + V(r)\right] \Psi_k(\vec{r}) = \dfrac{(\hbar \vec{k})^2}{2m} \Psi_k(\vec{r}),
\end{equation}
where $m$ is the reduced mass, $k$ is the wave number of the incident particle, and $V(r)$ is the interaction potential. Solutions to Eq. \eqref{eqn:Schrodinger} have at large distances from center of the potential the asymptotic form 
\begin{equation}  \label{eqn:asymptotic-solution}
    \Psi \approx \text{e}^{i\,k\,z} + \dfrac{f(\theta)}{r}\text{e}^{i\,k\,r}
\end{equation}
where the first describes the term describes the incident particle traveling in the positive direction of the z-axis. The second term in Eq. \eqref{eqn:asymptotic-solution} describes the scattered particles as a spherical wave emitted where $f(\theta)$ is some function of the scattering angle characteristic of the interaction potential. In Eq. \eqref{eqn:asymptotic-solution} $f(\theta)$ is often referred to as the scattering amplitude. The differential cross section corresponding to the solution to Eq. \eqref{eqn:Schrodinger} is then found by taking the ratio of the probability per unit time that the scattered particle pass through a surface element $dS = r^2 d\Omega$ to the current density of the incident wave, that is,
\begin{equation} \label{eqn:formula-distinguishable-particles}
    \sigma(E,\theta) = |f(\theta)|^2.
\end{equation}

Collisions where two identical particles collide requires special consideration. This is due to the identity of the particles making it impossible to say which of the particle is scattered and which is recoiling. The differential cross section that describes the interaction of identical particles is
\begin{equation} \label{eqn:formula-identical-particles}
    \sigma_{i}^{C} = | f(\theta) |^2 + | f(\pi - \theta) |^2 + \dfrac{(-1)^{2s}}{2s+1} \left[ f(\theta) f^{\star}(\pi-\theta) + f^{\star}(\theta) f(\pi-\theta) \right],
\end{equation}
where $s$ is the spin of the particles. The first term describes the forward scattering probability, the second term describes the backward scattering probability. The third term in Eq. \eqref{eqn:formula-identical-particles} is an interference term and characterizes the exchange interaction between the forward and backward directions.

In the remainder of this section the analytical forms of the unscreened and screened forms of the Coulomb differential cross section are derived from the scattering amplitudes for both distinguishable and identical particle collisions. Additionally, two different screening functions are introduced: 1) Moliere screening for cold solid materials, and 2) Debye screening for temperature dependent plasmas.

% ------------------------------------------------
% UNSCREENED COULOMB DIFFERENTIAL CROSS SECTIONS 
% ------------------------------------------------
\subsection{Unscreened Coulomb differential cross sections}
The un-screened interaction potential for two charged particles is described by the standard Coulomb potential,
\begin{equation} \label{eqn:Coulomb_potential}
    V(r) = \dfrac{Z_1 \, Z_2 \, e^2}{r},
\end{equation}
where $Z_1 e$ and $Z_2 e$ are the charges of the incident and target charged particles, and $r$ is the distance between the two charges. Note that the Coulomb potential allows for particles that are infinitely far apart to interact with one another, due to the $1/r$ term in Eq. \eqref{eqn:Coulomb_potential}. Solving Schrodinger's equation with the Coulomb potential yields the following scattering amplitude in Coulomb units,
\begin{equation} \label{eqn:coulomb-scattering-amplitude}
    f(\theta) = \dfrac{\gamma}{2 \, k \, \sin^2 \theta/2} \exp \left[ 2 \, i \, \gamma \, \log \sin \theta/2 \right] \dfrac{\Gamma(1 + i/k)}{\Gamma(1 - i/k)},
\end{equation}
where $\gamma = Z_1 \, Z_2 \, e^2 / \hbar \, v$, $v$ is the incident velocity, and $\hbar \, k = \mu \, v$. There are two possible cases for computing the differential cross section that corresponding to Eq. \eqref{eqn:coulomb-scattering-amplitude}: 1) the incident and target particles are distinguishable, and 2) the incident and target particles are identical. 

\subsubsection{Distinguishable particles}
Substituting Eq. \eqref{eqn:coulomb-scattering-amplitude} into Eq. \eqref{eqn:formula-distinguishable-particles} gives
\begin{equation} \label{eqn:dist_unscreened_coulomb}
    \sigma_{d}^{C} = \left(\dfrac{Z_1 \, Z_2 \, e^2}{2 \, \mu \, v^2}\right)^2 \dfrac{1}{\sin^4 \theta/2}
\end{equation}
where $\mu$ is the reduced mass of the system, and $v$ is the velocity of the incident particle. Eq. \eqref{eqn:dist_unscreened_coulomb} can is written in a more usable form by rewriting $e^2 = \alpha \, \hbar c$ and using the half angle trigonometry identity $2 \sin^2 \theta / 2 = 1 - \cos \theta$ to give
\begin{equation}
    \sigma_{d}^{C} = \left(\dfrac{Z_1 \, Z_2 \, \alpha \, \hbar c}{\mu \, v_1^2}\right)^2 \dfrac{1}{(1 - \mucm)^2}, \quad \text{where} \,\, \mucm = \cos \theta.
\end{equation}
Finally, expanding out the reduced mass $\mu = \dfrac{m_1 m_2}{m_1 + m_2}$, and using $m_1 \, v^2 = 2 \, E$ gives
\begin{equation} \label{eqn:coulomb-dcs}
    \sigma_{d}^{C}(E,\mucm) = \left(\dfrac{Z_1^2 \, Z_2^2 \, \left(\alpha \, \hbar c\right)^2}{ 4 \, E^2 \, \left(\frac{A}{1 + A} \right)^2}\right) \dfrac{1}{(1 - \mucm)^2}.
\end{equation}

Figure \ref{fig:sigma_c} shows a plot of Eq. \eqref{eqn:coulomb-dcs} for deuterons colliding with tritons at an incident energy of $1 \mev$ cutoff at $\mucm = 0.99$. Looking at Figure \ref{fig:sigma_c} the Coulomb differential cross section for distinguishable particles is highly peaked about $\mucm = 1$, meaning that particles will most likely undergo collisions that result in zero angular deflection/energy loss. The Coulomb differential cross section has a singularity as $\mucm \rightarrow 1$ due to the Coulomb interaction potential allowing for particles that are infinitely far apart to nonphysically interact with one another. As a result of the singularity Eq. \eqref{eqn:coulomb-dcs} cannot be integrated over the entire range of $\mucm$, and instead must be cutoff. 
\begin{figure}[!htb]
    \centering
    \includegraphics[]{../figures/interaction_physics/sigma-C.pdf}
    \caption{Distinguishable Coulomb differential cross section for energetic deuterons colliding with background tritons cutoff at $\mucm = 0.99$}
    \label{fig:sigma_c}
\end{figure}

The total scattering cross section associated with Eq. \eqref{eqn:coulomb-dcs} is found by integrating over $\mucm$ from $-1$ to some cutoff angle, $\mu_{cut}$ resulting in the following total scattering cross section
\begin{equation} \label{eqn:unscreened-total-xs}
    \sigma^C_d(E) = \left(\dfrac{Z_1^2 \, Z_2^2 \, \left(\alpha \, \hbar c\right)^2}{ 4 \, E^2 \, \left(\frac{A}{1 + A} \right)^2}\right) \dfrac{\mu_{cut}}{1-\mu_{cut}}, \quad \text{where} \,\, -1 < \mu_{cut} < 1.
\end{equation}
Figure \ref{fig:sigma_c_total} shows Eq. \eqref{eqn:unscreened-total-xs} for several different collisions and with a cutoff cosine of $\mu_{cut} = 0.999$. From Figure \ref{fig:sigma_c_total} as the energy of the incident particle increases the magnitude of the total cross section decreases, this can also be seen in Figure \ref{fig:sigma_c}. Additionally, as the mass of the incident particle increases so does the magnitude of the total cross section. Conversely, as the mass of the target particle increases, the magnitude of the of the total cross section decreases. Lastly, for nominal ICF energies, $<10$ MeV, the magnitude of Eq. \eqref{eqn:unscreened-total-xs} is large ($>1$ barn) meaning that for typical ICF densities the mean free path is very small and highly peaked about zero angular deflection.
\begin{figure}[!htb]
    \centering
    \includegraphics[scale=0.75]{../figures/interaction_physics/total-sigma-C.pdf}
    \caption{Total distinguishable Coulomb differential cross section for several collisions cutoff at $\mucm = 0.999$}
    \label{fig:sigma_c_total}
\end{figure}

\subsubsection{Identical particles}
Substituting Eq. \eqref{eqn:coulomb-scattering-amplitude} into Eq. \eqref{eqn:formula-identical-particles} gives the un-screened Coulomb differential cross section for identical particles, that is,
\begin{multline} \label{eqn:coulomb-unscreened-identical}
   \sigma_{i}^{C} = \dfrac{\gamma^2}{k^2} \left[ \dfrac{1}{(1 - \mucm)^2} + \dfrac{1}{(1 + \mucm)^2} \right. \\ \left. + \dfrac{(-1)^{2s}}{2s+1} \dfrac{2}{(1 - \mucm)(1 + \mucm)} \cos \left( \gamma \, \log \, \dfrac{1 - \mucm}{1 + \mucm} \right) \right].
\end{multline}
Figure \ref{fig:coulomb-identical} shows $\sigma_{C,i}$ for a variety of incident particle energies and cosine of CM scattering angles. From Figure \ref{fig:coulomb-identical} the identical particle differential cross section is highly peaked about $\mucm = -1, 1$ due to the Coulomb interaction potential. 
\begin{figure}[!htb]
    \centering
    \includegraphics[]{../figures/interaction_physics/sigma-Ci.pdf}
    \caption{The identical particle Coulomb differential cross section for deuterons colliding with background deuterons for $\mucm = [-0.99,0.99]$}
    \label{fig:coulomb-identical}
\end{figure}

Eq. \eqref{eqn:coulomb-unscreened-identical} can be greatly simplified when $\gamma \ll 1$ meaning that the velocity of the incident particle is so large that $\hbar v \gg Z_1 Z_2 e^2$. In this case the cosine in the third term can be replaced by unity and Eq. \eqref{eqn:coulomb-unscreened-identical} becomes
\begin{equation} \label{eqn:coulomb-unscreened-identical-simplified}
   \sigma_{i}^{C} \approx \dfrac{2 \, \gamma^2}{k^2 (1 - \mucm^2)} \left[\dfrac{1+ \mucm^2}{1-\mucm^2} + \dfrac{(-1)^{2s}}{2s+1}\right], \quad \text{when} \,\, \gamma \ll 1.
\end{equation}
For the opposite case, $\hbar v \ll Z_1 Z_2 e^2$, the cosine term in Eq. \eqref{eqn:coulomb-unscreened-identical} is a rapidly oscillating function, and the resulting cross section differs significantly from the Rutherford value given by Eq. \eqref{eqn:coulomb-unscreened-identical-simplified}. Figure \ref{fig:coulomb-identical-2d} shows the oscillations in the differential cross section due to the cosine term at various incident particle energies. In Figure \ref{fig:coulomb-identical-2d} it is clear that at low energies ($E < 10$ keV) these oscillations are significant; however, at higher energies these oscillations atr not significant and the Eq. \eqref{eqn:coulomb-unscreened-identical} rapidly approaches Eq. \eqref{eqn:coulomb-unscreened-identical-simplified}.

\begin{figure}[!htb]
    \centering
    \includegraphics[]{../figures/interaction_physics/sigma-Ci-2d.pdf}
    \caption{Comparison between the full (solid lines) and approximate (dashed lines) identical particle Coulomb differential cross sections at various energies for tritons colliding with background tritons for $\mucm = [-0.99,0.99]$.}
    \label{fig:coulomb-identical-2d}
\end{figure}

% ------------------------------------------------
% SCREENED COULOMB DIFFERENTIAL CROSS SECTIONS 
% ------------------------------------------------
\subsection{Screened coulomb differential cross sections}
To overcome the difficulties of the Coulomb interaction potential, namely the fact that it allows for particle infinitely far apart to interact with one another, Wentzel and Yukawa independently introduced the screened Coulomb potential,
\begin{equation} \label{eqn:wentzel-interaction-potential}
    V_s(r) = \dfrac{Z_1 \, Z_2 \, e^2}{r} \exp \left(-\frac{r}{R}\right),
\end{equation}
where $R$ is the characteristic screening length of the system. In Eq. \eqref{eqn:wentzel-interaction-potential} exponential term decreases more rapidly than the $1/r$ term thereby making the probability that particles will interact at distances greater than $R$ exponentially decreases with increasing interaction length $r$. This leads to an interaction potential that is finite as $r \rightarrow \infty$ while still have the same general behavior as the original Coulomb interaction potential.

The Schrodinger equation with the screened Coulomb interaction potential cannot be solved exactly and therefore its solution is approximated with the Born approximation. The Born approximation consists of approximating the the scattered wave function by a plane wave, and gives the following formula for the scattering amplitude
\begin{equation} \label{eqn:first-born-approximation}
    f(\theta) \approx - \dfrac{2 \, m}{\hbar^2} \int\limits_0^{\infty} U(r) \, \dfrac{\sin qr}{q} \, r \, dr
\end{equation}
where $q = 2 \, k \, \sin \theta / 2$. Substituting the screened interaction potential into Eq. \eqref{eqn:first-born-approximation} and yields
\begin{equation} \label{eqn:wentzel-scattering-amplitude}
    f_s(\theta) = \dfrac{Z_1 \, Z_2 \, e^2 \, m}{2 \, k^2 \, \hbar^2} \dfrac{1}{(2kR)^{-2} + \sin^2 \theta/2} = \dfrac{\gamma}{2 \, k} \left[\dfrac{1}{(2kR)^{-2} + \sin^2 \theta/2}\right].
\end{equation}
In Eq. \eqref{eqn:wentzel-scattering-amplitude} the term $(2kR)^{-2}$ is often referred to as the screening parameter and is denoted by $A_s$. Note that as $A_s \rightarrow 0$ Eq. \eqref{eqn:wentzel-scattering-amplitude} approaches the unscreened Coulomb potential scattering amplitude, Eq. \eqref{eqn:Coulomb_potential}, without the exponential and gamma function terms. In fact it has been shown that the Born approximation is not a valid approximation of the scattering amplitude when a screened interaction potential is used for nucleon-nucleon interactions unless the incident particles energy is at relativistic speeds. Nonetheless, the differential cross sections for distinguishable and identical particles are derived using Eq. \eqref{eqn:wentzel-scattering-amplitude} as they result in differential cross sections that are at least finite at $\mucm = 1$ and resemble the Coulomb differential cross sections previously derived.

Using Eqs. \eqref{eqn:formula-distinguishable-particles} and \eqref{eqn:formula-identical-particles} the screened Coulomb differential cross section for distinguishable and identical particles are
\begin{equation} \label{eqn:screened-coulomb-d}
    \sigma_{d}(E,\mucm) = \dfrac{\gamma^2}{k^2} \dfrac{1}{\left[A_s + 1 - \mucm\right]^2},
\end{equation}
and
\begin{multline} \label{eqn:screened-coulomb-i}
    \sigma_{i}(E,\mucm) = \dfrac{\gamma^2}{k^2} \left[ \dfrac{1}{\left(A_s + 1 - \mucm\right)^2} + \dfrac{1}{\left(A_s + 1 + \mucm\right)^2} \right. \\ \left. + \dfrac{(-1)^{2s}}{2s+1} \dfrac{2}{\left(A_s + 1 - \mucm\right)\left(A_s + 1 + \mucm\right) }\right].
\end{multline}
In Eq. \eqref{eqn:screened-coulomb-d} as $A_s \rightarrow 0$ it becomes the unscreened Coulomb differential cross sections (Eq. \eqref{eqn:coulomb-dcs}). However, in the identical particle case as $A_s \rightarrow 0$ instead of approaching the unscreened identical particle differential cross Eq. \eqref{eqn:screened-coulomb-i} becomes the approximate unscreened identical particle differential cross section (Eq. \eqref{eqn:coulomb-unscreened-identical-simplified}). This discrepancy is due to the validity of the Born approximation, which is only valid if the incident particle has high energy. However, Eqs. \eqref{eqn:screened-coulomb-d} and \eqref{eqn:screened-coulomb-i} are considered ``good enough'' approximations to the true differential cross section and are used throughout the remainder of this dissertation. 

% ------------------------------------------------
% SCREENING FUNCTIONS 
% ------------------------------------------------
\subsection{Screening Functions}
In this section the various screening functions for the Coulomb differential cross section that are available in LANDO are described. Additionally, we describe how to use a custom screening function. There are four screening functions provided in LANDO: none, constant, Moliere, and Debye. The following subsection discuss each of these functions.


\subsubsection{Moliere screening}
Moliere screening describes the screening of the nuclear charge of a nucleus due to the atomic electrons in the atom. The formula for Moliere screening is
\begin{equation}
    A_s^M(E) = \left( \dfrac{\hbar c}{2 \, pc \, a_I}\right)^2 \left[ 1.13 + 3.76  \, \left( \dfrac{\alpha \, z \, Z}{\beta}\right)\right]
\end{equation}
where $\hbar c$ is the reduced Planck constant; $\alpha$ is the fine structure constant; $pc$ is the momentum of the incoming particle; $z$ is the charge of the incoming particle; $Z$ is the charge of the target particle; $a_I$ is the universal screening length. The universal screening length is given by
\begin{equation}
    a_I = 
    \begin{cases}
        \quad \dfrac{C_{TF} \, a_0}{Z^{1/3}}, \quad\quad &z = 1 \\\\
        \dfrac{C_{TF} \, a_0}{z^{0.23} + Z^{0.23}}, \quad &z \geq 2
    \end{cases}, \quad
    C_{TF} = \dfrac{1}{2} \left( \dfrac{3 \, \pi}{4}\right)^{2/3}.
\end{equation}

Moliere screening can be used in LANDO by using the following screening function

\subsubsection{Debye screening}
Debye screening is a measure of a charge carrier's net electrostatic effect in a solution and how far its electrostatic effect persists. The Debye length is computed as
\begin{equation}
    \lambda_D = \left[ \epsilon_0 \sum_{j=0}^N \dfrac{k_B T_j}{z_j^2 \, n_j} \right]^{-1/2}
\end{equation}
where the index $j$ indicates the background species.

Debye screening can be used in LANDO by using the following screening function

\begin{figure}[!htb]
    \centering
    \begin{minipage}{.5\textwidth}
      \centering
      \includegraphics[width=\linewidth]{../figures/interaction_physics/moliere-screening.pdf}
      \captionof{figure}{A figure}
      \label{fig:test1}
    \end{minipage}%
    \begin{minipage}{.5\textwidth}
      \centering
      \includegraphics[width=\linewidth]{../figures/interaction_physics/debye-screening.pdf}
      \captionof{figure}{Another figure}
      \label{fig:test2}
    \end{minipage}
\end{figure}