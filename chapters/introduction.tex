\documentclass[../main.tex]{subfiles}

\begin{document}

\chapter{Introduction}
Charged particle transport (CPT) is an important problem in a vast array of applications, including: radiotherapy, space radiation, and plasma physics. The application of interest in this work is the transport of energetic light ions in inertial confinement fusion (ICF) plasmas; however, the work presented here can be easily applied to any application in which CPT is important. More specifically, this work focuses on the transport of energetic $\alpha$-particles in a spherical, fully-ionized, thermal deuterium-tritium (DT) plasma. This problem is commonly encountered in ICF applications where DT fusion reactions produce energetic $\alpha$-particles that subsequently transport and deposit energy in the DT plasma \cite{Bellan-2006}. In this work the following assumptions are made: all electric and magnetic effects present in the plasma are neglected, the plasma ions can only become energetic through recoil collisions, and the energetic light ions are absorbed upon reaching the thermal energy of the plasma.

In this proposal the CPT problem is divided into three areas: 1) the interaction physics that describe the collisions that light ions experience in plasma, 2) the mathematical models that describe the transport of the light ions in plasma, and 3) the numerical solution methods for solving the CPT equation. The remainder of this introduction provides brief overviews of the interaction physics, mathematical formulations, current numerical solution methods for the CPT equation, and the research goals of this work.

\section{Charged particle interaction physics overview}
Energetic light ions interact with the plasma through through three types of collisions: 1) Coulomb collisions, 2) nuclear-elastic scattering (NES) collisions, and 3) nuclear-reaction collisions \cite{Perkins-1981} \cite{Hale-1983}. Coulomb collisions are modeled by the Coulomb potential, which describes the interaction of two point charges that are a distance $r$ apart \cite{Bellan-2006}. Coulomb collisions are characterized by small angular-deflection and energy loss collisions, and small mean-free-paths (MFPs) or average distances between discrete collisions. As an example, for an energetic light ion to slow down to the thermal energy of the plasma via Coulomb collisions, it undergoes many small angular-deflection and energy-loss interactions with the background charged particles present in the plasma. These many small angular-deflection and energy collisions cause the light ions to very nearly continuously be scattered by the charged particles present in the plasma. These almost continuous collision events differ significantly from the discrete collision events experienced by neutral particles. 

NES and nuclear-reaction collisions are very discrete events and are not peaked about zero angular deflection and have large MFPs when compared to Coulomb collisions. The NES collisions are large-angle-elastic scattering collisions that result in significant amounts of energy being transferred from the energetic light ions to the plasma ions creating recoil ions. These recoil ions transport through the plasma creating additional recoil ions until the thermal energy of the plasma is reached. The nuclear-reaction collisions cause the incident ions to be absorbed and create new light ions of different species. These newly created light ions transport through the plasma and create additional recoil ions and cause further nuclear-reactions until reaching the thermal energy of the plasma. This cascading effect resulting from the NES and nuclear-reaction collisions causes the distributions in the plasma of all ion species to be coupled, meaning that the general CPT problem in plasmas is a fully-coupled problem.

\section{Mathematical models overview}
The CPT process describes both the free streaming of the light ions from one collision site to the next, and the collisions experienced by the light ions with the plasma. The general framework that describes this transport process is provided by the linear Boltzmann transport equation (BTE) \cite{Prinja-2010}. The BTE is an integro-differential equation that generally consists of seven independent variables: three in space, two in angular-directions, one in energy, and one in time. In this work, the simplified time-independent, one-dimensional spherical coordinates version of the general BTE is used. Additionally, the BTE collision operator is modified to account for the multi-species collisions that occur in CPT. The resulting time-independent, multi-species, one-dimensional spherical form of the BTE is referred to as the CPT equation.

Analytical solution methods to the CPT equation are not possible in most applications due to the complexities in the geometry and material cross sections that describe realistic problems. Instead, numerical solution techniques must be used for which two general approaches exist: Monte Carlo methods, and deterministic methods. Monte Carlo methods attempt to solve the CPT equation by using random numbers to track the transport of individual particles in the plasma, and deterministic methods discretize the CPT equation on a geometric mesh into a set of linear equations which are solved using linear algebra techniques. Traditional neutral particle Monte Carlo and deterministic solution methods are inefficient in obtaining accurate solutions to the CPT equation because of the small MFPs that are characteristic of light ions. In Monte Carlo methods, the small MFP causes the method to sample hundreds of thousands of individual collisions per light ion using random numbers. This large volume of random number sampling causes Monte Carlo methods to be very slowly converging. For a deterministic solution method to produce accurate solutions to a transport problem, the spacing between mesh points must be on the same order as the MFP \cite{Larsen-1999}, which in the context of CPT results in a large number of linear equations and unknowns that require significant computational time to solve. 

The inefficiencies exhibited by both Monte Carlo and deterministic solution methods are due to the Boltzmann collision operator, which is an efficient collision operator for the Coulomb collisions that dominate the interaction physics of light ions. Therefore more efficient collision operators are needed to more efficiently represent the interaction physics of light ions. Several collision operators that are more efficient than the Boltzmann collision operator at representing the interaction physics of light ions have been previously developed. The first is the Fokker-Planck (FP) collision operator, which results by retaining the first two terms in a Taylor series expansion of the Boltzmann collision operator about zero angular-deflection and energy-loss \cite{Pomraning-1996}. As a result of the Taylor series expansion, the FP collision operator is highly efficient at modeling problems in which the interaction physics are dominated by zero angular deflection and energy loss collisions. However, the FP collision operator does not account for large-angle scattering collisions which produce recoil ions. 

The second collision operator is the Boltzmann-Fokker-Planck (BFP) collision operator which accounts for the large-angle-scattering collisions by splitting the original Boltzmann collision operator into singular and smooth parts \cite{Ligou-1986}. The singular part describes the small angular-deflection and energy-loss collisions, while the smooth part describes the large-angle scattering collisions. The singular part is then modeled using an FP collision operator, while the smooth part is modeled with a Boltzmann collision operator. The challenge with BFP collision operators is properly splitting the original Boltzmann collision operator such that the peaked part is sufficiently peaked so that the asymptotic limit of the FP scattering is satisfied, and the smooth part is smooth enough that Boltzmann collision operators can be efficiently solved \cite{Pomraning-1992}. 

\section{Numerical solution methods overview}
Several efficient Monte Carlo solution methods have been developed for numerical solutions to the CPT equation, including the condensed history method \cite{Berger-1963}, a hybrid multigroup/continuous energy BFP method \cite{Morel-1996}, and a moment preserving method \cite{Dixon-2015}. However, the focus of this work is on deterministic solution methods and subsequently does not consider the development of efficient Monte Carlo solution methods. 

Several deterministic solution methods have been previously developed. The first is a finite element BFP method developed by Honrubia and Aragones in 1986 \cite{Honrubia-1986}. Honrubia and Aragones developed a fully coupled space-energy-angle linear discontinuous Galerkin discretization for the BFP equation in one-dimensional spherical geometry. The main limitations of that work are: a cutoff angle was used to split the Boltzmann collision operator into peaked and smooth parts, and NES and nuclear-reaction collisions were not taken into account. By using a cutoff angle, a very poor representation of the peaked and smooth parts of the Boltzmann collision operator is obtained \cite{Landesman-1989}. Neglecting NES and nuclear-reaction collisions  means that no recoil ions are produced which have been shown to be a significant source for the heating of the plasma \cite{Nakao-1988}.

The second deterministic method for solving the CPT equation was developed by Nakao, et al. in 1990 \cite{Nakao-1990}. Nakao, et al. developed a fully coupled space-energy one-dimensional spherical linear discontinuous Galerkin BFP solver. Nakao, et al. split the original Boltzmann collision operator by modeling the Coulomb collisions with an FP collision operator, and modeling the NES collision with a Boltzmann collision operator. There were several shortcomings of this work: the first being the failure to include the energy diffusion term in the FP collision operator. Second, better tabulated data that describes the NES and nuclear-reaction collisions is now available \cite{Hale-1983}. 

\section{Research goals}
The goal of this work is to provide accurate and efficient solution methods for the general CPT transport equation that describes the transport of energetic-light ions in spherical ICF plasmas. To achieve this goal, the current methods for CPT are introduced and research is proposed in all three areas of CPT outlined in this introduction. In the charged particle interaction physics area of CPT, the research that is proposed is the development and proper utilization of analytical and tabulated differential cross sections that describe the interactions of light ions in plasma. This includes the exploration of more accurate Coulomb differential cross sections and screening parameters in addition to the proper utilization of the tabulated NES and nuclear-reaction data by use of two-body collision kinematics.

To improve upon the limitations of the FP and BFP collision operators, the generalized Fokker-Planck (GFP) collision operator is introduced. The GFP collision operator re-normalizes the Taylor series expansion of the Boltzmann collision operator such that a robust set of coupled second order in angle, and first order in energy, differential equations are obtained. As a result of this renormalization, the GFP method preserves the first few terms of the Taylor series expansion of the Boltzmann collision operator exactly and approximates the remaining higher-order moments in terms of the first 2K moments. By retaining these higher-order moments, the GFP collision operator is able to account for larger angular scattering collision than the FP collision operator which completely neglects them.

To accurately solve the CPT equation, the various streaming and collision operators that appear in the CPT equation are discretized using the discrete ordinates \Sn method in angle and an arbitrary order discontinuous Galerkin discretization method in space and energy. These discretizations are presented and verified using the method of manufactured solution in this proposal. Additionally, to numerically solve the CPT equation efficiently, iterative solution methods are proposed for both solving the fully coupled CPT equation and for the transport equation describing each individual ion species.

\section{Dissertation outline}
Section 2 of this proposal describes the interaction physics of charged particles in detail and introduces the CPT equation. Section 3 provides the details of the FP and the BFP collision operators including a discussion of the properties and limitations of each. In Section 4, the numerical discretization of the various operators encountered in CPT are introduced using the discrete ordinates and discontinuous Galerkin methods. Section 5 describes the current state of the charged particle data, including: the current analytical form of the Coulomb differential cross section, and how the multigroup data required by the numerical methods is computed. Finally, in Section 6 the proposed research required to meet the research goal of this work is presented. 
\end{document}
