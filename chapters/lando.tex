\documentclass[../main.tex]{subfiles}

\begin{document}
\chapter{The Charged Particle Data Processing Library LANDO}
LANDO is a C++ charged particle data processing code that produces JSON formatted multigroup charged particle data from ENDF files. Figure \ref{fig:LANDO_code_structure} shows the general code structure and data processing method that LANDO uses; in general LANDO first reconstructs the charged particle cross sections using the \textbf{Construct} class and a \textit{screening} function. Next LANDO interprets the constructed cross sections into light/heavy reactions using the \textbf{Interpret} class using the Boltzmann-Fokker-Planck (BFP) \textit{decomposition} functions. Finally, LANDO generates the multigroup data from the interpretation by using the \textbf{Multigroup} class and a \textit{weighting} function.

\begin{figure}[!htb]
\centering
\resizebox{0.5\columnwidth}{!}{%
\includegraphics{../figs/landoDesign.pdf}
}
\caption{Flow chart describing the general structure of LANDO}
\label{fig:LANDO_code_structure}
\end{figure}

The remainder of this documentation describes the classes \textbf{Construct}, \textbf{Interpret}, and \textbf{Multigroup}, and the auxiliary functions needed by these classes. First the classes are described in detail, and then the auxiliary functions are discussed.

\section{Construct class}
The construct class in LANDO reconstructs all of the cross sections available in an ENDF file according to the ENDF manual. This is done for both the total cross sections (MF3) and the differential cross sections (MF6). The class is templated on a labmda function that serves as the screening function for the Coulomb differential cross section. For example, to construct the cross sections for deuterium-tritium system from the availible ENDF file, the Construct class would be constructed as:

\begin{minted}[frame=lines, framesep=2mm, baselinestretch=1.2, bgcolor=LightGray, fontsize=\footnotesize, linenos]{cpp}
 #include "Lando.hpp"

 auto screening = lando::screening::none();
 lando::Construct<decltype(screening)> data("d-001_H_002.endf", screening);
\end{minted}

Once an ENDF file has been reconstructed, the following methods are available to the following return sub-classes:

\begin{minted}[frame=lines, framesep=2mm, baselinestretch=1.2, bgcolor=LightGray, fontsize=\footnotesize, linenos]{cpp}
 using construct = lando::Construct<decltype(screening)>;
  
 Construct::MF1 mf1 = data.generalInformation();
 std::vector<construct::MF3> mf3 = data.reactionCrossSections();
 std::vector<construct::MF6> mf6 = data.productEnergyAngleDistributions();
\end{minted}

The following subsections describe each of the sub-classes, and the methods within each sub-class to access the information and reconstructed cross sections.

\subsection{General information}
The general information class provides the same general information about the reaction as the original ENDF file. The following methods are available:
\begin{minted}[frame=lines, framesep=2mm, baselinestretch=1.2, bgcolor=LightGray, fontsize=\footnotesize, linenos]{cpp}
 auto  ZA = mf1.ZA();
 auto ZAI = mf1.ZAI();
  
 auto AWR = mf1.AWR();
 auto AWR = mf1.materialAtomicMassRatio();

 auto LRP = mf1.LRP();
 auto LRP = mf1.resonanceParameterFlag();

 auto LFI = mf1.LFI();
 auto LFI = mf1.isFissile();

 auto NLIB = mf1.NLIB();
 auto NLIB = mf1.libraryType();

 auto NMOD = mf1.NMOD();
 auto NMOD = mf1.modificationNumber();

 auto ELIS = mf1.ELIS();
 auto ELIS = mf1.excitationEnergy();

 auto STA = mf1.STA();
 auto STA = mf1.isStable();

 auto LIS = mf1.LIS();
 auto LIS = mf1.excitedLevel();

 auto LISO = mf1.LISO();
 auto LISO = mf1.isomericLevel();

 auto NFOR = mf1.NFOR();
 auto NFOR = mf1.libraryFormat();

 auto AWI = mf1.AWI();
 auto AWI = mf1.projectileAtomicMassRatio();

 auto EMAX = mf1.EMAX();
 auto EMAX = mf1.maximumEnergy();

 auto LREL = mf1.LREL();
 auto LREL = mf1.releaseNumber();

 auto NSUB = mf1.NSUB();
 auto NSUB = mf1.subLibrary();

 auto NVER = mf1.NVER();
 auto NVER = mf1.versionNumber();

 auto TEMP = mf1.TEMP();
 auto TEMP = mf1.temperature();

 auto RTOL = mf1.RTOL();
 auto RTOL = mf1.reconstructionTolerance();

 auto LDRV = mf1.LDRV();
 auto LDRV = mf1.derivedMaterial();
\end{minted}


\subsection{Reaction cross sections}
The reaction cross section provides the angle-integrated reaction cross sections for each reaction in an ENDF file. The following general information is provided for each reaction:
\begin{minted}[frame=lines, framesep=2mm, baselinestretch=1.2, bgcolor=LightGray, fontsize=\footnotesize, linenos]{cpp}
 auto reaction = mf3[0];

 auto energies = reaction.energies();
 auto values = reaction.values();
 auto interpolations = reaction.interpolations();
 auto interpolationBounds = reaction.interpolationBounds();
 auto QI = reaction.QI();
 auto QM = reaction.QM();

 auto MT = reaction.MT();
 auto ZA = reaction.ZA();
 auto AWR = reaction.AWR();
 auto ZAI = reaction.ZAI();
 auto AWI = reaction.AWI();
\end{minted}

This class also provides a method to access the angle-integrated reaction cross section that takes an energy in MeV and returns the cross section in barns, see the following example:
\begin{minted}[frame=lines, framesep=2mm, baselinestretch=1.2, bgcolor=LightGray, fontsize=\footnotesize, linenos]{cpp}
  double E = 1.0; // MeV
  auto sigma = reaction.sigma(E); // barns
\end{minted}

\subsection{Energy-angle distributions}
The \textbf{Energy-angle distributions} sub-class reconstructs the differential cross sections for File 6 in the ENDF files. This sub-class requires information from both the \textbf{Information} and \textbf{Reaction cross sections} sub-classes to be able to reconstruct the differential cross sections. The differential cross section for any reaction are defined in the ENDF files by giving the production cross section for each reaction product in barns/steradian assuming azimuthal symmetry:
\begin{equation} \label{eqn:differential-cross-section}
    \sigma_i(E^{\prime},E,\mu_0) = \sigma(E^{\prime}) \, y_i(E^{\prime}) \, f_i(E^{\prime},E,\mu_0) / 2\, \pi
\end{equation}
where:
\begin{align*}
    i \quad & \text{denotes one particular product} \\
    E^{\prime} \quad & \text{is the incident energy} \\
    E \quad & \text{is the energy of the product emitted with cosine of scattering $\mu_0$} \\
    \sigma(E^{\prime}) \quad & \text{is the reaction cross section} \\
    y_i(E^{\prime}) \quad & \text{is the product yield or multiplicity for particle $i$} \\
    f_i \quad & \text{is the normalized distribution with units $(\text{eV}\cdot \text{unit-cosine})^{-1}$ }
\end{align*}

For each possible reaction (MT) that exists in the ENDF file, LANDO reconstructs the differential cross section for all products. The product differential cross section are represented in the ENDF using seven different representations (law's in ENDF), they are:
\begin{itemize}
    \item LAW 0: Unknown distribution
    \item LAW 1: Continuum energy-angle distribution
    \item LAW 2: Discrete two-body scattering
    \item LAW 3: Isotropic discrete emission
    \item LAW 4: Discrete two-body recoils
    \item LAW 5: Charged particle elastic scattering
    \item LAW 6: N-body phase space
    \item LAW 7: Laboratory angle-energy
\end{itemize}
Hence, each reactions possible differential cross sections are tabulated as a vector of a variant of all seven distribution types. The following code snippet demonstrates how to access a specific reactions differential cross sections:

\begin{minted}[frame=lines, framesep=2mm, baselinestretch=1.2, bgcolor=LightGray, fontsize=\footnotesize, linenos]{cpp}
 #include "Lando.hpp"

 /* Construct the cross sections */
 auto screening = lando::screening::none();
 lando::Construct<decltype(screening)> data("d-001_H_002.endf", screening);
 
 /* Define te various LAWS */
 using LAW0 = decltype(data)::MF6::Unknown;
 using LAW1 = decltype(data)::MF6::ContinuumEnergyAngle;
 using LAW2 = decltype(data)::MF6::DiscreteTwoBodyScattering;
 using LAW3 = decltype(data)::MF6::IsotropicDiscreteEmission;
 using LAW4 = decltype(data)::MF6::DiscreteTwoBodyRecoils;
 using LAW5 = decltype(data)::MF6::ChargedParticleElasticScattering;
 using LAW6 = decltype(data)::MF6::NBodyPhaseSpace;
 using LAW7 = decltype(data)::MF6::LaboratoryAngleEnergy;
 
 /* Define the type of the products in MF6 */
 using Product = std::variant< LAW0, LAW1, LAW2, LAW3, LAW4, LAW5, LAW6, LAW7 >;
 
 /* Get the energy-angle distributions */                  
 std::vector< decltype(data)::MF6 > differentialCrossSections 
    = data.productEnergyAngleDistributions();
    
 /* Get a specific reaction */
 std::vector< Product > crossSection = differentialCrossSections[0];
\end{minted}

To access a specific cross section within the vector of distributions, one should use \textbf{std::visit} in combination with \textbf{njoy::utility::overload} where lambda functions for each of the types in the previous code snippet are defined. The following code snippet demonstrates this
\begin{minted}[frame=lines, framesep=2mm, baselinestretch=1.2, bgcolor=LightGray, fontsize=\footnotesize, linenos]{cpp}
 /* Starting with crossSection from previous code snippet */ 
 std::visit( njoy::utility::overload{ 
        [] (const LAW0& ) {},
        [] (const LAW1& ) {},
        [] (const LAW2& ) {},
        [] (const LAW3& ) {}, 
        [] (const LAW4& ) {},
        [] (const LAW5& ) {}, 
        [] (const LAW6& ) {},
        [] (const LAW7& ) {} },
    crossSection[0] );
\end{minted}

Every cross section has the following base methods available to help identify the specific the reaction occuring:
\begin{minted}[frame=lines, framesep=2mm, baselinestretch=1.2, bgcolor=LightGray, fontsize=\footnotesize, linenos]{cpp}
 /* Starting with a specific cross section from previous code snippet */
 LAW crossSection = std::get< LAW >(differentialCrossSections[0]); // LAW is 0-7
 
 auto  MT = crossSection.MT(); // ENDF reaction identifier

 auto ZAI = crossSection.ZAI(); // ZIAD of incident particle
 auto  ZA = crossSection.ZA();  // ZIAD of target particle 
 auto ZAP = crossSection.ZAP(); // ZIAD of emitted particle

 auto AWI = crossSection.AWI(); // Weight of incident particle
 auto AWR = crossSection.AWR(); // Weight of target particle
 auto AWP = crossSection.AWP(); // Weight of emitted particle
  
 auto LIP = crossSection.LIP(); // ENDF product modifier
 auto LAW = crossSection.LAW(); // Distribution type of cross section
\end{minted}

Additionally, for every distribution the method \textbf{sigma} is provided which returns the value of the differential cross at the energies angle specified. For two-body interactions, \textbf{sigma} is only a function of the incident energy in the LAB frame and cosine of the CM scattering angle; for multi-body reactions, \textbf{sigma} is a function of the incident energy in the LAB frame, cosine of the CM scattering angle for the product, and the outgoing energy of the product in the LAB frame.
\begin{minted}[frame=lines, framesep=2mm, baselinestretch=1.2, bgcolor=LightGray, fontsize=\footnotesize, linenos]{cpp}
 /* Starting with a cross section from previous code snippet */
 auto value1 = crossSection.sigma(5.0, 0.25); // Two-body reaction
 
 auto value2 = crossSection.sigma(5.0, 4.0, 0.25); // Multi-body reaction
\end{minted}
In the remainder of this section the methods by which each of the seven possible distribution types are used to reconstruct the normailizied distributions in Eq. \eqref{eqn:differential-cross-section} are discussed. 

% \subsection{LAW 0: Unknown}
% Not implemented.

% \subsection{LAW 1: Continuum energy angle}
% This distribution is used to describe particles emitted in multi-body reactions or combinations of several reactions, such as scattering through a range of levels or reactions at high energies where many channels are normally open.

% \subsubsection{Legendre coefficients representations}
% Legendre coefficients are used as follows:
% \begin{equation}
%     f(E^{\prime},E,\mu_0) = \sum_{l=0}^{NA} \dfrac{2l + 1}{2} \, f_l(E^{\prime},E) \, P_l(\mu_0)
% \end{equation}
% where NA is the number of angular parameters. 

% \subsubsection{Kalbach-Mann systematics representation}
% Kalbach-Mann systematics are a powerful representation for outgoing neutron and charged particle energy-angle distributions in high energy (usually pre-equilibrium) reactions. The Kalbach-Mann formulation addresses reactions of the form 
% \begin{equation}
%     a + A = C = B + b
% \end{equation}
% where:
% \begin{align*}
%     a \quad &\text{is the incident projectile}, \\
%     A \quad &\text{is the target nucleus}, \\
%     C \quad &\text{is the compound nucleus}, \\
%     b \quad &\text{is the emitted particle}, \\
%     B \quad &\text{s the residual nucleus}.
% \end{align*}

% The double differential distribution for projectiles other than photons is
% \begin{equation}
%     f(E_a,E_b^C,\mu_0) = \dfrac{a \, f_0}{2 \, \sinh{a}} \Big[ \cosh \left(a \, \mu_0\right) + r \, \sinh \left(a \, \mu_0\right) \Big]
% \end{equation}
% where:
% \begin{align*}
%     E_a   \quad &\text{is the energy of the incident projectile a in the laboratory system}, \\
%     E_b^C \quad &\text{is the energy of the emitted particle in the CM system}, \\
%     \mu_0 \quad &\text{is the cosine of the scattering angle of the emitted particle b in the CM system}, \\
%     a     \quad &\text{is a simple parameterized function }, \\
%     f_0   \quad &\text{is the total emission probability}, \\
%     r     \quad &\text{is the pre-compound fraction as given by the evaluator.} \\
% \end{align*}

% The formula for calculating the slope value $a = a(E_a, E_b^C)$ is:
% \begin{equation}
%     a = C_1 \, X_1 + C_2 \, X_1^3 + C_3 \, M_a \, m_b \, X_3^4
% \end{equation}
% where:
% \begin{equation*}
%     \begin{split}
%         e_a &= \epsilon_a + S_a \\
%         R_1 &= \min(e_a, E_{t1}) \\
%         X_1 &= R_1 \, e_b / e_a
%     \end{split}
%     \quad\quad\quad\quad\quad
%     \begin{split}
%         e_b &= \epsilon_b + S_b \\
%         R_3 &= \min(e_a, E_{t3}) \\
%         X_3 &= R_3 \, e_b / e_a
%     \end{split}
% \end{equation*}
% The parameter values for light particle induced reactions are:
% \begin{equation*}
%     \begin{split}
%         \epsilon_a &= E_a \left(\dfrac{\text{AWR}_A}{\text{AWR}_A + \text{AWR}_a}\right) \\
%         C_1 &= 0.04 / \text{MeV} \\
%         C_3 &= 6.7 \cdot 10^{-7} / (\text{MeV})^4 \\
%         E_{t1} &= 130 \text{MeV} \\
%         M_n &= 1 \\
%         M_d &= 1 \\
%         m_n &= 1/2 \\
%         m_d &= 1 \\
%         m_{He-3} &= 1
%     \end{split}
%     \quad\quad\quad\quad\quad
%     \begin{split}
%         \epsilon_b &= E_b^C \left(\dfrac{\text{AWR}_B + \text{AWR}_b}{\text{AWR}_B} \right)\\
%         C_1 &= 1.8 \cdot 10^{-6} / (\text{MeV})^3 \\
%         \\
%         E_{t3} &= 41 \text{MeV} \\
%         M_p &= 1 \\
%         M_{\alpha} &= 0 \\
%         m_p &= 1 \\
%         m_t &= 1 \\
%         m_{\alpha} &= 2
%     \end{split}
% \end{equation*}
% Note that Kalbach never specified $M_t$ or $M_{He-3}$ as the systematics have not been extended to high energy t and 3He projectiles, where the $X_3$ term contributes.

% $S_a$ and $S_b$ are the separation energies for the incident and emitted particles, respectively, for the reaction $A + a \rightarrow C \rightarrow B + b$. As the particle emission is occuring for relatively high energies, pairing and shell effects in the compound nucleus are muted and must be neglected in the calculation of separation energies. Therefore, one should use the following formulae for the separation energies3 in MeV:
% \begin{multline}
%     S_a = 15.68 \left[ A_C - A_A \right] - 28.07 \left[ \dfrac{(N_C - Z_C)^2}{A_C} - \dfrac{(N_A - Z_A)^2}{A_A} \right] \\
%     - 18.56 \left[ A_C^{2/3} - A_A^{2/3} \right] + 33.22 \left[ \dfrac{(N_C - Z_C)^2}{A_C^{4/3}} - \dfrac{(N_A - Z_A)^2}{A_A^{4/3}} \right] \\
%     - 0.717 \left[ \dfrac{Z_C^2}{A_C^{1/3}} - \dfrac{Z_A^2}{A_A^{1/3}} \right] + 1.211 \left[ \dfrac{Z_C^2}{A_C} - \dfrac{Z_A^2}{A_A} \right] - I_a
% \end{multline}
% and
% \begin{multline}
%     S_a = 15.68 \left[ A_C - A_B \right] - 28.07 \left[ \dfrac{(N_C - Z_C)^2}{A_C} - \dfrac{(N_B - Z_B)^2}{A_B} \right] \\
%     - 18.56 \left[ A_C^{2/3} - A_B^{2/3} \right] + 33.22 \left[ \dfrac{(N_C - Z_C)^2}{A_C^{4/3}} - \dfrac{(N_B - Z_B)^2}{A_B^{4/3}} \right] \\
%     - 0.717 \left[ \dfrac{Z_C^2}{A_C^{1/3}} - \dfrac{Z_B^2}{A_B^{1/3}} \right] + 1.211 \left[ \dfrac{Z_C^2}{A_C} - \dfrac{Z_B^2}{A_B} \right] - I_b
% \end{multline}
% where:
% \begin{align*}
%     A,B,C   \quad &\text{subscripts for the target, residual, and compound nuclei respectively}, \\
%     N, Z, A  \quad &\text{are the neutron, proton, and mass numbers of each nuclei}, \\
%     I_a, I_b \quad &\text{are the energies required to separate the incident and emitted particles}.
% \end{align*}

% \subsection{LAW 2: Discrete two-body scattering}
% This law is used to describe the distribution in energy and angle of particles described by two-body kinematics. Since the energy of a particle emitted with a particular scattering cosine $\mu_0$ is determined by kinematics (see chapter 1), it is only necessary to give:
% \begin{equation} \label{eqn:law2-legendre}
%     p_i(E^{\prime}, \mu_0) = \int dE \, f_i(E^{\prime},E,\mu_0) = \dfrac{1}{2}+ \sum\limits_{\ell = 1}^L \dfrac{2\ell + 1}{2} \, a_{\ell}(E{\prime}) \, P_{\ell}(\mu_0)
% \end{equation}
% where the $P_{\ell}$ are the Legendre polynomials with the maximum order $L$, and $a_{\ell}(E)$ are the Legendre coefficients at energy $E$. In ENDF files the data is given as seta of $E_i^{\prime}$ and $a_{i,\ell}$ values. To reconstruct the differential cross section: one first find the appropriate energy bin and reconstructs the values $(E_i^{\prime}, p_i)$ and $(E_{i+1}^{\prime}, p_{i+1})$ using the Eq. \eqref{eqn:law2-legendre}. Then one interpolates for the value $(E^{\prime}, p)$ using the appropriate interpolation technique as defined by the reaction (see Chapter 2 for interpolation methods).

% \subsection{LAW 3: Isotropic discrete emission}
% The cross section is the same as LAW 2 except there is no angular distribution. Therefore the differential cross section is simply:
% \begin{equation}
%     \sigma_i(E^{\prime}) = \sigma(E^{\prime}) \, y_i(E^{\prime})
% \end{equation}

% \subsection{LAW 4: Discrete two-body recoils}
% If the recoil nucleus of a two-body reaction does not break up, its energy and angular distribution can be determined from the kinematics. For these type of reactions ENDF files contain no LAW dependent structure, and it is upto the user to reconstruct these differential cross sections. This is straight forward to do because the cosine of the CM recoil angle, $\eta_0$, is simply equal to minus the cosine of the CM scattering angle $\mu_0$; therefore these cross section are simply
% \begin{equation}
%     f_{LAW 4}(E^{\prime},\eta_0) = f_{LAW 2}(E^{\prime}, -\mu_0).
% \end{equation}

% \subsection{LAW 5: Charged particle elastic scattering}
% Elastic scattering of charged particles includes components from Coulomb scattering, nuclear scattering, and the interference between them. The screened Coulomb differential cross sections for distiguishible and identical particles are derived in Chapter 3. The total differential cross section is written as
% \begin{equation}
%     \sigma_{el}(E^{\prime},\mu_0) = \sigma_{C}(E^{\prime},\mu_0) + \sigma_{N}(E^{\prime},\mu_0) + \sigma_{I}(E^{\prime},\mu_0),
% \end{equation}
% where the subscripts $C$, $N$, and $I$ represent the Coulomb, nuclear-elastic scattering, and interference parts of the differential cross section.

% \subsection{LAW 6: N-body phase space}
% In the absence of detailed information, it is often useful to use n-body phase-space distributions for the particles emitted from neutron and charged-particle reactions. These distributions conserve energy and momentum, and they provide reasonable kinematic limits for secondary energy and angle in the LAB system. Note, LAW 6 is meant for situations where the number of emitted particles is $\geq 3$. In the CM system, the phase-space distribution for particle i is
% \begin{equation}
%     p_i^{CM}(E^{\prime},E,\mu_0) = C_n \, \sqrt{E} \, \left(E_i^{\text{max}} - E\right)^{(3n/2)-4}
% \end{equation}
% where $E_i^{\text{max}}$ is the maximum possible CM energy for particle $i$, and $C_n$ are the normalization coefficients
% \begin{align*}
%     C_3 &= \dfrac{4}{\pi \, (E_i^{\text{max}})^2}\\
%     C_4 &= \dfrac{105}{32 \, (E_i^{\text{max}})^{7/2}} \\
%     C_5 &= \dfrac{256}{14 \, \pi \, (E_i^{\text{max}})^5}
% \end{align*}

% The value of $E_i^{\text{max}}$ is a fraction of the energy availible i CM,
% \begin{equation}
%     E_i^{\text{max}} = \dfrac{M - m_i}{M} E_a
% \end{equation}
% where $M$ is the total mass of the $n$ particles being treated by the law.

% \subsection{LAW 7: Laboratory angle-energy}
% Not implemented.

\section{Interpret Class}
In this section the \textbf{Interpret} class is explained in detail. The \textbf{Interpret} class serves two purposes: 1) to translate the cross sections built by the construct class into light and heavy reactions, and 2) to decompose the charged particle elastic scattering differential cross section into peaked and smooth parts. Here light reactions refer to reactions where the incident particle is of less or equal mass to the target, and heavy reactions refer to reactions where the incident particle is heavier than the target particle. As previously mentioned in the kinematics chapter, in ENDF files data is only available for light reactions, and therefore to get the data required for heavy reactions the frame of reference must be transformed from the incident particle to the target particle using kinematics. The collision can then be handled in the incident particle's frame of reference as a light reaction for which data is available.

The decomposition methods used by this class are provided externally to the \textbf{Interpret} class, and are located in the namespace \emph{lando::decomposition}. Two functions must be provided: 1) a function describing the smooth part of the differential cross section, and 2) a function describing the peaked portion of the differential cross section. Several such decomposition's are provided in LANDO and are described in the \emph{Boltzmann-Fokker-Planck decomposition functions} section of this chapter. This section only serves to describe how these functions are utilized and not what they are.

The \textbf{Interpret} class is constructed in the following manner:
\begin{minted}[frame=lines, framesep=2mm, baselinestretch=1.2, bgcolor=LightGray, fontsize=\footnotesize, linenos]{cpp}
 #include "Lando.hpp"

 /* Construct Cross Sections */
 auto screening = lando::screening::moliere();
 lando::Construct<decltype(screening)> data("d-001_H_003.endf", screening);

 /* Construct Interpretation */
 auto smooth = lando::decomposition::fokker_planck::smooth();
 auto peaked = lando::decomposition::fokker_planck::peaked();
 lando::Interpret<decltype(data), decltype(smooth), decltype(peaked)> 
    interpretation( data, smooth, peaked );
\end{minted}
The \textbf{Interpret} class has two main member functions to access the cross section available, the following code demonstrates these functions
\begin{minted}[frame=lines, framesep=2mm, baselinestretch=1.2, bgcolor=LightGray, fontsize=\footnotesize, linenos]{cpp}
 auto reactionCrossSections = interpretation.reactionCrossSections();
 auto energyAngleDistributions = interpretation.energyAngleDistributions();
\end{minted}

\subsection{Reaction cross sections}

\subsection{Energy-angle distributions}

\section{Multigroup Class}
In this chapter the details of how the multigroup charged particle data is computed for use by the CPT methods.

\subsection{Group Constants}

\subsection{Transfer Matrices}
Consider the definition of the group to group transfer
\begin{equation} \label{eqn:g2g_transfer}
  \int\limits_{E_{g-\frac{1}{2}}}^{E_{g+\frac{1}{2}}} dE \int\limits_{4 \, \pi} d\hat{\Omega}^{\prime} \int\limits_{E_{g^{\prime}-\frac{1}{2}}}^{E_{g^{\prime}+\frac{1}{2}}} dE^{\prime} \,\, \Sigma(E^{\prime}\rightarrow E; \hat{\Omega}^{\prime} \rightarrow \hat{\Omega}) \,\, \Psi(E^{\prime},\hat{\Omega}^{\prime})
\end{equation}
By assuming that the background is homogeneous and isotropic, the scattering becomes a function of the angle between the initial and final directions, $\hat{\Omega}^{\prime}$ and $\hat{\Omega}$. The cosine of the scattering angle, $\mucm$ , can be written in terms of the initial and final scattering angles as
\begin{equation}
 \mucm = \hat{\Omega}^{\prime} \cdot \hat{\Omega} = \mu^{\prime} \mu + \sqrt{1-\left(\mu^{\prime}\right)^2} \sqrt{1-\mu^{2}} \cos\left(\phi^{\prime} - \phi \right) 
\end{equation}
and in general
\begin{equation} \label{eqn:mucm_dxs}
  \Sigma(E^{\prime}\rightarrow E; \hat{\Omega}^{\prime} \rightarrow \hat{\Omega}) = \Sigma(E^{\prime}\rightarrow E; \mucm).
\end{equation}

To convert this into a function of initial and final directions, Eq. \eqref{eqn:mucm_dxs} is expanded in a series of legendre polynomials
\begin{equation} \label{eqn:legendre_expansion_dxs}
  \Sigma(E^{\prime}\rightarrow E; \mucm) = \sum_{l=0}^L \dfrac{2l+1}{2} \, \Sigma_l(E^{\prime}\rightarrow E) \, P_l(\mucm)
\end{equation}
where
\begin{equation}
  \Sigma_l(E^{\prime}\rightarrow E) = \int\limits_{-1}^{1} d\mucm \, P_l(\mu) \, \Sigma(E^{\prime}\rightarrow E; \mucm).
\end{equation}
Substituting Eq. \eqref{eqn:legendre_expansion_dxs} back into Eq. \eqref{eqn:g2g_transfer} and using the relationship,
\begin{equation}
  P_l(\mucm) = \sum\limits_{m=-l}^l Y_{n}^{m}(\hat{\Omega}^{\prime}) \, Y_{n}^{m}(\hat{\Omega}),
\end{equation}
gives
\begin{equation}
  \sum_{l=0}^{\infty} \dfrac{2l+1}{2} \sum\limits_{m=-l}^l Y_{n}^{m}(\hat{\Omega}) \int\limits_{E_{g-\frac{1}{2}}}^{E_{g+\frac{1}{2}}} dE \int\limits_{4 \, \pi} d\hat{\Omega}^{\prime} \, Y_{n}^{m}(\hat{\Omega}^{\prime}) \int\limits_{E_{g^{\prime}-\frac{1}{2}}}^{E_{g^{\prime}+\frac{1}{2}}} dE^{\prime} \, \Sigma_l(E^{\prime}\rightarrow E) \, \Psi(E^{\prime},\hat{\Omega}^{\prime}).
\end{equation}
Next, defining the spherical harmonic moment of the flux as
\begin{equation}
  \Phi_l^m(E^{\prime}) = \int_{4\,\pi} d\hat{\Omega}^{\prime} \,\, Y_{n}^{m}(\hat{\Omega}^{\prime}) \,\, \Psi(E^{\prime},\hat{\Omega}^{\prime}),
\end{equation}
gives the final form of the expression as
\begin{equation} \label{eqn:transfer-matrix-full}
  \sum_{l=0}^{\infty} \dfrac{2l+1}{2} \sum\limits_{m=-l}^l Y_{n}^{m}(\hat{\Omega}) \int\limits_{E_{g-\frac{1}{2}}}^{E_{g+\frac{1}{2}}} dE \int\limits_{E_{g^{\prime}-\frac{1}{2}}}^{E_{g^{\prime}+\frac{1}{2}}} dE^{\prime} \, \Sigma_l(E^{\prime}\rightarrow E) \, \Phi_l^m(E^{\prime}).
\end{equation}
which expresses the transfer to the final direction $\hat{\Omega}$, in terms of the spherical harmonics moments of the flux. Finally, in the azimuthally symmetric system Eq. \eqref{eqn:transfer-matrix-full} can be simplified by integrating over the azimuthal angle; in this case, all the terms will be zero except those with $m = 0$, giving
\begin{equation} \label{eqn:transfer-matrix-simple}
  \sum_{l=0}^{\infty} \dfrac{2l+1}{2}  P_{l}(\mu) \int\limits_{E_{g-\frac{1}{2}}}^{E_{g+\frac{1}{2}}} dE \int\limits_{E_{g^{\prime}-\frac{1}{2}}}^{E_{g^{\prime}+\frac{1}{2}}} dE^{\prime} \, \Sigma_l(E^{\prime}\rightarrow E) \, \Phi_l(E^{\prime})
\end{equation}
where
\begin{equation}
  \Phi_l(E) = \int\limits_{-1}^{1} d\mu \, P_l(\mu) \, \Psi(E,\mu).
\end{equation}

From Eq. \eqref{eqn:transfer-matrix-simple} the group constants are defined as
\begin{equation} \label{eqn:group-constants}
  \sigma_{g^{\prime}\rightarrow g,l} = \int\limits_{E_{g-\frac{1}{2}}}^{E_{g+\frac{1}{2}}} dE \int\limits_{E_{g^{\prime}-\frac{1}{2}}}^{E_{g^{\prime}+\frac{1}{2}}} dE^{\prime} \Phi_l(E^{\prime}) \int\limits_{-1}^{1} d\mucm \, P_l(\mu) \, \Sigma(E^{\prime}\rightarrow E; \mucm).
\end{equation}

Before proceeding to the evaluation methods for Eq. \eqref{eqn:group-constants} we will first express the secondary-energy-angle distribution in the form that it is represented in evaluated libraries, where each reaction is represented in the form
\begin{equation}
  \Sigma(E^{\prime}\rightarrow E; \mucm) = m(E^{\prime}) \, \Sigma_s(E^{\prime}) \, f(E^{\prime},E,\mucm)
\end{equation}
where
\begin{itemize}
  \item $m(E^{\prime})$ is the multiplicity for the reaction
  \item $\Sigma(E^{\prime})$ total cross section for the reaction
  \item $p(E^{\prime},\mucm)$ angular distribution, which is a function of the incident energy. For correlated distributions, it is usually given in the center-of-mass system. Uncorrelated distributions are given in the laboratory system. Either may be given in terms of Legendre coefficients
  or tabulated values.
  \item $g(\mucm, E^{\prime}\rightarrow E)$ Energy distribution: For correlated distributions, this is an implied Dirac delta function, which expresses the exact correlation between scattering cosine and initial and final energies. For uncorrelated distributions, it is the actual secondary-energy
  distribution, independent of $\mucm$.
\end{itemize}


\subsubsection{Correlated distributions}
In the case of correlated distributions, the secondary distribution is expressed as the product of the angular distribution and a Dirac delta function expressing the exact correlation between scattering angle and initial and final energies, as
\begin{equation}
  f(E^{\prime},E,\mucm) = p(E^{\prime},\mucm) \, \delta\left[\mucm - S(E^{\prime},E)\right]
\end{equation}

In this case the Eq. \eqref{eqn:group-constants} can be written as
\begin{equation}
  \sigma_{g^{\prime}\rightarrow g,l} = \int\limits_{E_{g^{\prime}-\frac{1}{2}}}^{E_{g^{\prime}+\frac{1}{2}}} dE^{\prime} \, w(E^{\prime}) \, F_l(E^{\prime})
\end{equation}
where
\begin{equation}
  F_l(E^{\prime}) = \int\limits_{-1}^{1} d\mucm \, P_l(\mu) \int\limits_{E_{g-\frac{1}{2}}}^{E_{g+\frac{1}{2}}} dE \, p(E^{\prime},\mucm) \, \delta\left[\mucm - S(E^{\prime},E)\right].
\end{equation}
Finally, this integral is rewritten in the laboratory system as
\begin{equation}
  F_l(E^{\prime}) = \int\limits_{\mu_{min}}^{\mu_{max}} p(E^{\prime},\mucm[\mu]) \, P_l(\mu) \left| \dfrac{d\mucm}{d\mu} \right|d\mu 
\end{equation}
In these equations, the cosine limits are a function of the incident energy $E^{\prime}$ and correspond to the minimum and maximum scattering cosines that will permit a particle with an initial energy $E^{\prime}$ to end up with a secondary energy between $E_{g-\frac{1}{2}}$ and $E_{g+\frac{1}{2}}$.

For inelastic scattering the relationship between the incident energy, outgoing energy, and the cosine of the laboratory scattering angle is
\begin{equation}
  \mu = \dfrac{1}{2} \left[ (A+1) \sqrt{\frac{E}{E^{\prime}}} - (A-1) \sqrt{\frac{E^{\prime}}{E}} + \dfrac{AQ}{E} \right]
\end{equation}

For elastic scattering the relationship between the incident energy, outgoing energy, and the cosine of the laboratory scattering angle is
\begin{equation}
  \mu = \dfrac{1}{2} \left[ (A+1) \sqrt{\frac{E}{E^{\prime}}} - (A-1) \sqrt{\frac{E^{\prime}}{E}} \right]
\end{equation}

\subsubsection{Uncorrelated distributions}

\section{Boltzmann Fokker-Planck Decomposition Functions}
This section describes the various Boltzmann-Fokker-Planck (BFP) decomposition functions available in LANDO. The BFP decomposition functions serve to separate the total elastic scattering differential cross section into peaked and smooth parts as:
\begin{equation}
    \sigma(E^{\prime},\mu_0) = \sigma_P(E^{\prime},\mu_0) + \sigma_S(E^{\prime},\mu_0), 
\end{equation}
where the subscripts $P$ and $S$ represent the smooth and peaked parts. The peaked part is then handled with a Fokker-Planck expansion, while the smooth part is dealt with using the typical transfer matrix.

The following BFP decomposition are provided by LANDO: Fokker-Planck, Boltzmann, nuclear-elastic scattering, exponential fit, partial-range fit, and a moment-based method. The remainder of this section describes the implementation details of the BFP decomposition's available in LANDO.

\subsection{Boltzmann}
If no BFP decomposition is desired, the decomposition functions that should be selected are the Boltzmann decomposition methods. These functions set the smooth part of the decomposition equal to the total differential cross section and peaked part equal to zero, that is,
\begin{equation}
    \sigma_P(E^{\prime},\mu_0) = 0, \quad \quad \sigma_S(E^{\prime},\mu_0) = \sigma(E^{\prime},\mu_0).
\end{equation}

The following code snippet demonstrates the use of the Boltzmann decomposition functions:
\begin{minted}[frame=lines, framesep=2mm, baselinestretch=1.2, bgcolor=LightGray, fontsize=\footnotesize, linenos]{cpp}
 #include "Lando.hpp"

 /* Construct Cross Sections */
 auto screening = lando::screening::moliere();
 lando::Construct<decltype(screening)> data("d-001_H_002.endf", screening);
 
 /* Construct Interpretation */
 auto smooth = lando::decomposition::boltzmann::smooth();
 auto peaked = lando::decomposition::boltzmann::peaked();
 lando::Interpret<decltype(data), decltype(smooth), decltype(peaked)> 
    interpretation( data, smooth, peaked );
\end{minted}

\subsection{Fokker-Planck}
The Fokker-Planck BFP decomposition functions set the entire peaked cross section equal to the total elastic scattering differential cross section and the smooth cross section equal to zero, that is,
\begin{equation}
    \sigma_S(E^{\prime},\mu_0) = 0, \quad \quad \sigma_P(E^{\prime},\mu_0) = \sigma(E^{\prime},\mu_0).
\end{equation}

The following code snippet demonstrates the use of the Boltzmann decomposition functions:
\begin{minted}[frame=lines, framesep=2mm, baselinestretch=1.2, bgcolor=LightGray, fontsize=\footnotesize, linenos]{cpp}
 #include "Lando.hpp"

 /* Construct Cross Sections */
 auto screening = lando::screening::moliere();
 lando::Construct<decltype(screening)> data("d-001_H_002.endf", screening);
 
 /* Construct Interpretation */
 auto smooth = lando::decomposition::fokker_planck::smooth();
 auto peaked = lando::decomposition::fokker_planck::peaked();
 lando::Interpret<decltype(data), decltype(smooth), decltype(peaked)> 
    interpretation( data, smooth, peaked );
\end{minted}

\subsection{Nuclear-elastic scattering}
The nuclear-elastic scattering (NES) BFP decomposition functions separate the total elastic scattering into smooth and peaked parts by setting the smooth part equal to NES differential cross section and the peaked part equal to the Coulomb. plus interference pieces, that is,
\begin{equation}
    \sigma_S(E^{\prime},\mu_0) = \sigma_{N}(E^{\prime},\mu_0), \quad \quad \sigma_P(E^{\prime},\mu_0) = \sigma_{CI}(E^{\prime},\mu_0).
\end{equation}

The following code snippet demonstrates the use of the Boltzmann decomposition functions:
\begin{minted}[frame=lines, framesep=2mm, baselinestretch=1.2, bgcolor=LightGray, fontsize=\footnotesize, linenos]{cpp}
 #include "Lando.hpp"

 /* Construct Cross Sections */
 auto screening = lando::screening::moliere();
 lando::Construct<decltype(screening)> data("d-001_H_002.endf", screening);
 
 /* Construct Interpretation */
 auto smooth = lando::decomposition::nuclear_elastic_scattering::smooth();
 auto peaked = lando::decomposition::nuclear_elastic_scattering::peaked();
 lando::Interpret<decltype(data), decltype(smooth), decltype(peaked)> 
    interpretation( data, smooth, peaked );
\end{minted}

\subsection{Moment based}

\section{Examples}
\end{document}